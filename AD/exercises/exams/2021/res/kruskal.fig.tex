\begin{figure}
  \centering
  \begin{tikzpicture}[auto,on grid, node distance=2cm, vertex/.style={circle, draw, minimum size=0.75cm}]
    \node[vertex] (a) {a};
    \node[vertex, right=of a] (b) {b};
    \node[vertex, right=of b] (c) {c};
    \node[vertex, right=of c] (d) {d};
    \node[vertex, right=of d] (e) {e};
    \node[vertex, below right=2cm and 1cm of a] (f) {f};
    \node[vertex, right=of f] (g) {g};
    \node[vertex, right=of g] (h) {h};
    \node[vertex, right=of h] (i) {i};

    \draw (a) -- node {17} (b) -- node {31} (c) -- node {5} (d) -- node {23} (e) -- node {13} (i) -- node {19} (h) -- node {7} (g) -- node {23} (f) -- node {29} (a);
    \draw (b) -- node {3} (f);
    \draw (c) -- node {11} (g);
    \draw (c) -- node {13} (h);
    \draw (d) -- node {2} (h);
  \end{tikzpicture}

  \caption{Ein Beispielgraph für die Anwendung des Algorithmus von Kruskal.}\label{fig:kruskal2021}
\end{figure}