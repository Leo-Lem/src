\documentclass{exercisesheet}

\subject{Algorithmen \& Datenstrukturen}
\semester{Wintersemester 2023}
\author{Leopold Lemmermann}
% \withsolutions

\begin{document}

\createtitle

\exercisesheet[1]{Präsenzaufgaben}
\setcounter{section}{0}
\begin{eexercises}{Vollständige Induktion}{
    Beweisen Sie die folgenden Aussagen mit vollständiger Induktion:
  }
  \item $\sum_{k=1}^{n}{k} = \frac{n(n+1)}{2}\forall n \in \mathbb{Z}_{\geq 0}$.
  \item $\sum_{k=0}^{n-1} = 2^n-1\forall n \in \mathbb{Z}_{\geq 0}$.
\end{eexercises}

\begin{exercise}{Algorithmenanalyse}
  Was berechnet der Algorithmus Alg? Finden Sie einen Aufruf für den der Algorithmus besonders langsam ist. Wie verhält sich der Algorithmus bei Eingabe von rationalen oder reelen Zahlen?
  \begin{algorithm}[ht]
    \caption{Alg}
    \KwData{$a,b \in \mathbb{Z}_{>0}$}
    \KwResult{$c \in \mathbb{Z}_{>0}$}
    \If{$a=b$}{\Return{$a$}}
    \If{$a<b$}{\Return{$\text{Alg}(b-a,a)$}}
    \If{$b<a$}{\Return{$\text{Alg}(a-b,b)$}}
  \end{algorithm}
\end{exercise}

\begin{exercise}{Algorithmenentwurf}
  Entwerfen Sie einen Algorithmus, der für gegebene ganze Zahlen $a_1, \ldots, a_n \in \mathbb{Z}$ das Minimum und das Maximum bestimmt. Versuchen Sie dabei möglichst wenige Vergleiche zu verwenden ($\leq 1,5n$ sind möglich).
\end{exercise}

\newpage\setcounter{section}{1}\setcounter{subsection}{0}
\begin{eexercises}{Big-O Notation}{
    Beweisen oder widerlegen Sie die folgenden Aussagen:}
  \item $2n = \bigO{n}$
  \item $n^2 = \bigO{n}$
  \item $n \log n = \bigO{n^2}$
  \item $n^2 + n = \bigO{n^2}$
\end{eexercises}

\begin{exercise}{Algorithmus BALLSELECTION}
  Betrachten Sie den folgenden, umgangssprachlich formulierten Algorithmus. Gegeben sei eine Urne mit einer geraden Anzahl $n = 2k - 1$ weißer Kugeln und einer beliebigen Anzahl $m \geq 1$ schwarzer Kugeln.\par
  \begin{algorithm}[ht]
    \caption{BALLSELECTION}
    \While{wenigstens zwei Kugeln in der Urne sind}{
      Nimm zwei beliebige Kugeln aus der Urne heraus.\\
      \If{beide Kugeln die gleiche Farbe haben}{
        Entferne die Kugeln aus dem Spiel und lege eine neue, schwarze Kugel in die Urne.
      }
      \Else{
        Lege die weiße Kugel in die Urne zurück und entferne lediglich die schwarze Kugel aus dem Spiel.
      }
    }
  \end{algorithm}
  \noindent Beweisen Sie, dass die letzte Kugel in der Urne schwarz ist. Geben Sie dazu eine geeignete Schleifeninvariante an.
\end{exercise}

\newpage\setcounter{section}{2}\setcounter{subsection}{0}
\begin{eexercises}{Erweiterte Big-O Notation}{
    Beweisen oder widerlegen Sie die folgenden Aussagen:
  }
  \item $2n = \Omega(n)$
  \item $\log_a(n+1) = \Theta(\log_b(n+1))$ für $a,b \in \mathbb{R}_{>1}$.
  \item $n = o(2n)$
\end{eexercises}

\begin{exercise}{Mastertheorem}\setcounter{subsection}{0}
  Bestimmen Sie die Größenordnung der Funktionen, wenn möglich mittels Mastertheorem. Falls das Mastertheorem nicht anwendbar sein sollte, begründen Sie dies und verwenden stattdessen die Substitutionsmethode.
  \hint{Es ist hier kein Induktionsbeweis erforderlich bei Anwendung der Substitutionsmethode.}
  \begin{enumerate}
    \item $T_1(n) = \begin{cases}
              1                               & \text{falls } n = 1 \\
              16 \cdot T_1(\frac{n}{2}) + n^2 & \text{sonst}
            \end{cases}$
    \item $T_2(n) = \begin{cases}
              1                              & \text{falls } n = 1 \\
              8 \cdot T_2(\frac{n}{2}) + n^4 & \text{sonst}
            \end{cases}$
    \item $T_3(n) = \begin{cases}
              0                    & \text{falls } n = 0 \\
              3 \cdot T_3(n-1) + 2 & \text{sonst}
            \end{cases}$
  \end{enumerate}
\end{exercise}

\newpage\setcounter{section}{3}\setcounter{subsection}{0}
\begin{eexercises}{Sortieralgorithmen}{
    Beantworten Sie die folgenden Fragen:
  }
  \item Sortieren Sie das folgende Feld mit Radixsort. Die Basis ist 8 und die Zahlen sind zur Basis 8 angegeben.
  \begin{align*}
    142, 204, 154, 7, 104, 162, 521, 17, 262, 504, 370, 252
  \end{align*}
  \item Argumentieren Sie, dass $n$ natürliche Zahlen aus dem Bereich $\{0, 1, \ldots, n \cdot c - 1\}$ für eine Konstante $c \in \mathbb{N}$ mit Laufzeit $\bigO{n}$ sortiert werden können.
\end{eexercises}

\begin{exercise}{HEAPSORT}
  Stellen Sie das Verfahren von HEAPSORT angewendet auf die Instanz $[1,5,6,7,8,7,9,9,10]$ grafisch dar.
\end{exercise}

\newpage\setcounter{section}{4}\setcounter{subsection}{0}
\begin{exercise}{Rot-Schwarz-Bäume}
  Führen Sie die folgenden Operationen (einfügen bzw. löschen von Schlüsseln) mit einem initial leeren Rot-Schwarz-Baum aus. Geben Sie den Baum nach jedem Einfügen bzw. Löschen sowie nach jeder Rotation bzw. Umfärbung an.
  \begin{enumerate}
    \item Insert(40), Insert(48), Insert(68), Insert(55), Insert(39), Delete(40), Delete(48)
  \end{enumerate}
\end{exercise}

\begin{eexercises}{Hashtabellen}{
    Konstruieren Sie die Hashtabelle der Größe $m = 23$, die durch Einfügen der Elemente
    \begin{align*}
      47, 17, 24, 70, 22, 01, 40, 45, 36, 59
    \end{align*}
    mit der nachfolgenden Methode entsteht:
  }
  \item Divisionsmethode,
  \item Multiplikationsmethode mit $c = 1/2$ und
  \item erweiterter Divisionsmethode mit $a = 5$ und $b = 3$
\end{eexercises}

\newpage\setcounter{section}{5}\setcounter{subsection}{0}
\begin{exercise}{Topologische Sortierung}
  Finden Sie für den unten gezeigten Graph aus Abb. 1 eine Topologische Sortierung. Ist es möglich mehrere verschiedene Lösungen zu finden?
  \begin{figure}[ht]
    % TODO: add graph
  \end{figure}
\end{exercise}

\begin{exercise}{Kruskals Algorithmus}
  Bestimmen Sie mit Kruskals Algorithmus aus der Vorlesung einen minimalen Spannbaum für den Graphen aus Abb. 2. Machen Sie die Reihenfolge in der die Kanten abgearbeitet wurden deutlich.
  \begin{figure}[ht]
    % TODO: add graph
  \end{figure}
\end{exercise}

\newpage\setcounter{section}{6}\setcounter{subsection}{0}
\begin{exercise}{Independent Set}
  Ein Independent Set in einem Graphen $G = (V, E)$ ist eine Menge von Knoten $I$, so dass für je zwei Knoten $i, j \in I$ gilt, dass $\{i, j\} \notin E$. Für das Problem INDEPENDENT SET ist ein Graph $G = (V, E)$ sowie eine Zahl $k \in \mathbb{N}$ gegeben und es soll entschieden werden, ob es in $G$ ein Independent Set mit $k$ Knoten gibt.\par
  Zeigen Sie, dass das Problem INDEPENDENT SET NP-vollständig ist, indem Sie eine Reduktion von CLIQUE auf INDEPENDENT SET angeben.
\end{exercise}

\begin{exercise}{Textumbruch}
  Gegeben seien Wörter $w_1, \ldots, w_n$. Diese Wörter sollen auf Textzeilen aufgeteilt werden. Dazu seien $c(i, j)$ die Kosten, um Wörter die $w_i, \ldots, w_j$ in eine Textzeile zu schreiben für alle $1 \leq i \leq j \leq n$.\par
  Geben Sie mithilfe dynamischer Programmierung einen Algorithmus an, der Linebreaks $l_1, \ldots, l_m$ so setzt, sodass $c(1,l_1)+c(l_1 +1,l_2)+\ldots+c(l_m +1,n)$ minimiert wird.
\end{exercise}

\exercisesheet[2]{Hausaufgaben}
\setcounter{section}{0}

% Aufgabe 1.1 Sei F die Menge der Funktionen N ! R+ und seien e, f, g, h 2 F. Beweisen oder widerlegen Sie die folgenden Aussagen:
% 1. Ause=O(f)undg=O(h)folgt(e·g)=O(f·h)
% 2. Ause=O(f)undg=O(h)folgt(e+g)=O(f+h)
% 3. Ause=O(f)undg=O(h)folgt(e+g)=O(max(f,h))
% Hierbei sind ·, +, max : F ⇥ F ! F punktweise aufzufassen, also ist zum Beispiel (e · g) die durch (e · g)(n) = e(n) · g(n) definierte Funktion.
% Aufgabe 1.2 Ordnen Sie die über die folgenden Ausdrücke definierten Funktionen fi : N>0 ! R0, i = 1,...,10 der Größe nach im Sinne der O-Notation: f1(n) := n32 , f2(n) := loglogn, f3(n) := nn, f4(n) := pn, f5(n) := n log n, f6(n) := n, f7(n) := log2 n, f8(n) := 2n, f9(n) := n",für 0 < " < 12, f10(n) := logn.
% Aufgabe 1.3 Beweisen oder widerlegen Sie die folgenden Aussagen: 1. n3 = O(3n2),
% 2. n+logn=O(n·logn),
% 3. n5 = O(2log2 n),
% 4. 3n = O(2n)
% Aufgabe 1.4 Die folgende Funktion erhält ein Eingabearray aus natürlichen Zahlen als Eingabe. • Wenden Sie den Algorithmus auf ein Beispielarray der Länge mindestens drei an.
% • Geben Sie an, was der Algorithmus berechnet.
% • Bestimmen Sie eine geeignete Schleifeninvariante.
% • Beweisen Sie mit Hilfe der Schleifeninvariante die Korrektheit des Algorithmus.
% Algorithm 1: Alg
% Input :array A[i],i = 1,2,...,n, A[i] 2 N
% Output:result 2 R>0
% 1 result=0
% 2 fori=1tondo
% 3 result = ((i  1) · result + A[i])/i
% 4 returnresult

% Aufgabe 2.1 Sei F die Menge der Funktionen N ! R+ und seien f,g 2 F. Beweisen oder widerlegen Sie die folgenden Aussagen:
% 1. Mitf(n)=n2 2ngiltf(n)=⇥(n2)
% 2. Aus f(n) = O(g(n)) folgt g(n) = ⌦(f(n))
% Aufgabe 2.2 Bestimmen Sie die Größenordnung der Funktionen wenn möglich mittels Mas- tertheorem. Falls das Mastertheorem nicht anwendbar sein sollte, begründen Sie dies und verwenden stattdessen die Substitutionsmethode. Hinweis: Es ist hier kein Induktionsbeweis erfor- derlich bei Anwendung der Substitutionsmethode.
% T1(n) = (1 falls n = 1
% 4 · T1(dn/2e) + n2  sonst
% T2(n) = (1 falls n = 1
% 2·T2(n1)+4 sonst
% T3(n) = (1  falls n = 1
% 42 · T3(dn/3e) + 39 · T3(bn/3c) + 5n3 + 69n sonst

% Aufgabe 3.1 Eine zyklische einfach verkettete Liste ist eine einfach verkettete Liste, bei der zu- sätzlich der next-Zeiger des letzten Elements auf das erste Element der Liste zeigt. Im Folgenden seien außerdem die Elemente der Liste sortiert (aufsteigend oder absteigend, wählen sie selbst).
% 1. Geben Sie für die Operationen INSERT und DELETE einer solchen sortierten zyklischen Liste Pseudocode an. Achten Sie darauf, dass die Elemente nach den Operationen sortiert bleiben und dass ihre Operationen O(n) Zeit benötigen. (Achten Sie auf Randfälle!)
% 2. Begründen Sie, warum ihre Implementationen aus Punkt 1 O(n) Zeit benötigen.
% Aufgabe 3.2
% 1. Sortieren Sie das Array (12, 10, 6, 3, 1, 14, 9) mithilfe des QUICKSORT-Algorithmus aus der Vorlesung. Wählen Sie hierzu das Pivotelement wie im QUICKSORT-Algorithmus aus der Vorlesung vorgegeben. Stellen Sie den Inhalt des Arrays nach jedem Aufruf von PARTITION dar.
% 2. Welchen Einfluss auf die Laufzeit hat allgemein die Auswahl des Pivotelements bei QUICK- SORT?
% Skizzieren Sie worst-case- und best-case-Eingaben für eine konkrete Auswahl des Pivot- elements (z.B. immer am linken oder rechten Rand). Begründen Sie Ihre Antwort.
% 3. Ist der QUICKSORT-Algorithmus aus der Vorlesung stabil? Begründen Sie Ihre Antwort. Aufgabe 3.3 Zeigen Sie per struktureller Induktion, dass in jedem (binären) Heap H für die
% Anzahl Blätter b(H) und Anzahl Nicht-Blätter n(H) gilt: n(H)  b(H)  n(H)+1
% Aufgabe3.4InderVorlesungwurdederSELECT-Algorithmusbesprochen,derdasEingabearray 8Punkte in 5er-Gruppen aufteilt. Der Algorithmus kann aber auch leicht für andere Gruppengrößen
% angepasst werden. Sie können (wie in der Vorlesung) davon ausgehen, dass die Elemente des Eingabearrays paarweise verschieden sind.
% 1. Untersuchen Sie ob die Argumente für die Laufzeitabschätzung auch für 7er-Gruppen funktionieren. Finden und begründen Sie dafür eine entsprechende Rekurrenzgleichung und schätzen Sie die Laufzeit ab.
% 2. Finden und begründen Sie eine entsprechende Rekurrenzgleichung für den Fall mit 3er- Gruppen. Geben Sie zudem eine kurze Einschätzung dazu an, ob in diesem Fall auch eine lineare Laufzeit erreicht wird (hier ist kein formaler Beweis gefordert).
% 3. StellenSieeineRekurrenzgleichungfüreineVariantevonQUICKSORTauf,diedenSELECT- Algorithmus für die Wahl des Pivotelements verwendet und schätzen Sie die Laufzeit mit dem Mastertheorem ab

% Aufgabe 4.1 Im Folgenden betrachten wir Zahlen n = 2k  1 mit k 2 N>0.
% 1. In welcher Reihenfolge sollten die Zahlen 1, 2, . . . , n in einen initial leeren Suchbaum eingefügt werden, damit dieser möglichst balanciert/niedrig ist? Begründen Sie Ihre Antwort kurz.
% 2. Geben Sie Pseudocode für ein Verfahren mit O(n) Laufzeit an, dass einen entsprechenden Suchbaum erzeugt. Geben Sie eine kurze Begründung an, weshalb ihr Verfahren korrekt ist und die entsprechende Laufzeit erreicht.
% 3. Entwerfen Sie einen Algorithmus, der in linearer Laufzeit einen beliebigen Suchbaum mit n Einträgen in einen balancierten Suchbaum umwandelt. Begründen Sie kurz Laufzeit und Korrektheit des Verfahrens. Pseudocode muss nicht angegeben werden. (Hinweis: Betrachten Sie Theorem 1 und 2 zu Suchbäumen aus der Vorlesung.)
% Aufgabe 4.2
% 1. Wir betrachten Hashing mit offener Adressierung und quadratischer Sondierung, also eine Hashfunktion der Form h(k, i) = (h0(k) + c1i + c2i2) mod m. Es seien h0(x) = x mod 13, sowie c1 = c2 = 12 . Fügen Sie die Elemente 938, 1243, 10026, 71, 831, 555, 142, 768, 301, 176, 9347, 32418 und 360 in der angegebenen Reihenfolge in eine Hashtabelle der Größe m = 13 ein. Stellen Sie das Verfahren graphisch dar. Dabei sollte klar werden wo Kollisionen auftreten und wie diese aufgelöst werden.
% 2. Was passiert, wenn als letzte Zahl anstatt der 360 eine 359 eingefügt wird? Aufgabe 4.3 Es sei G = (V, E) ein ungerichteter Graph. Zeigen Sie:
% 1. Es gilt Pv2V deg(v) = 2|E|.
% 2. Wenn G zusammenhängend (verbunden) ist, dann gilt |E|  |V |  1. 3. Wenn G azyklisch ist, dann gilt |E|  |V |  1.
% 4. WennGeinBaumist,danngilt|E|=|V|1.
% 5. Bäume sind kantenmaximal kreisfrei und kantenminimal zusammenhängend. Wird also eine Kante hinzugefügt bzw. entfernt, so ist der entsprechende Graph nicht mehr kreisfrei bzw. zusammenhängend.
% Aufgabe 4.4 Gegeben seien zwei Rot-Schwarz-Bäume B1 und B2 und ein Element x 2 Z, sodass fürallex1 2B1 undx2 2B2 gilt:x1.keyxx2.key.
% 1. Beschreiben Sie einen Algorithmus, welcher aus der Vereinigung B = B1 [ {x} [ B2 einen neuen Rot-Schwarz-Baum in O(log n) Zeit berechnet, wobei n die Gesamtanzahl aller Knoten in B1 und B2 ist.
% 2. Begründen Sie, warum eurer Algorithmus die Laufzeitschranke einhält.

% Aufgabe 5.1
% 6 Punkte
%                      Abbildung 1: Graph G
% Betrachten Sie obigen Graphen G aus Abb. 1. Wenden Sie jeweils das verlangte Verfahren an
% bzw. beantworten Sie die Frage oder begründen Sie, warum dies nicht geht.
% 1. Ermitteln Sie mit der Breitensuche einen Breitensuchbaum (BFS-Baum). Starten Sie den
% Algorithmus bei s.
% 2. Ist das Ergebnis der Breitensuche eindeutig?
% 3. Ermitteln Sie mit der Tiefensuche einen Tiefensuchwald und insb. die Zeiten u.d und u.f für jeden Knoten. Starten Sie den Algorithmus wieder bei s.
% 4. Geben Sie eine topologische Sortierung des Graphen G an.
% 5. Bestimmen Sie mit dem Algorithmus aus der Vorlesung die starken Zusammenhangskom- ponenten von G. Geben Sie dazu den transponierten Graph und den Komponentengraph an.
% Aufgabe 5.2
% 6 Punkte
%  Abbildung 2: Graph G
% 1. Betrachten Sie den Graphen in Abb. 2. Bestimmen Sie mit Hilfe des Algorithmus von Prim einen minimalen Spannbaum des Graphen und skizzieren diesen. Der Startknoten ist grau gefärbt. Geben Sie zusätzlich die Reihenfolge an, in der die Kanten des Spannbaums gemäß Algorithmus hinzugefügt werden.
% 2. Sei G = (V, E) ein zusammenhängender gewichteter Graph mit Gewichtsfunktion w : E ! R und T = (V, E0) ein minimaler Spannbaum von G.
% (a) Für eine beliebige Kante e 2 E \ E0 wird deren Gewicht w(e) erhöht. Wir bezeichnen diesen modifizierten Graphen als Gˆ. Zeigen Sie, dass T ebenfalls ein minimaler Spannbaum von Gˆ ist.
% (b) Für eine beliebige Kante e 2 E wird deren Gewicht w(e) verringert. Wir bezeichnen diesen modifizierten Graphen als Gˆ. Konstruieren Sie einen Algorithmus, welcher einen minimalen Spannbaum von Gˆ mit Hilfe von T in Zeit O|V | + |E| berechnet. Hinweis: Hier ist kein Pseudocode gefordert.

% Aufgabe 5.3
% 6 Punkte
% 3 10 8 9
% 4 2 6 11 13
% 1 5 7 12 14 15
% Abbildung 3: Graph G
%                  Eine Kante e 2 E eines zusammenhängenden, ungerichteten Graphen G = (V, E) heißt Brückenkante, falls der Graph G0 = (V, E\e), der durch das Entfernen der Kante e aus G entsteht, nicht mehr zusammenhängend ist.
% 1. Bestimmen Sie alle Brückenkanten im oben abgebildeten Graphen aus Abb. 3.
% 2. Geben Sie einen Algorithmus in Pseudocode an, der in Laufzeit O(|V | · |E| + |E|2) ent- scheidet, ob ein zusammenhängender, ungerichteter Graph G eine Brückenkante besitzt oder nicht.
% 3. Zeigen Sie, dass Ihr Algorithmus korrekt arbeitet.
% 4. Zeigen Sie, dass Ihr Algorithmus die Laufzeitschranke einhält.

% Aufgabe 5.4
% 6 Punkte
% s
% 12 3 A1B2C
% 1 4F
% 38 3t
%         42 D4E
% 23
%     Abbildung 4: Graph G
% 1. Wenden Sie Dijkstras Algorithmus auf den Graphen aus Abb. 4 an. Beginnen Sie bei Knoten s. Geben Sie tabellarisch die am Ende jeder Iteration der While-Schleife in der Queue enthaltenen Keys an.
% 2. Modifizieren Sie den Pseudocode von Dijkstras Algorithmus so, dass er als Input einen Graphen, einen Startknoten s und einen Zielknoten v als Input nimmt und als Output die explizite Darstellung des kürzesten Pfades von s nach v als Sequenz von Knoten ausgibt. Begründen Sie kurz die Korrektheit Ihrer Lösung.
% Die Laufzeit des modifizierten Algorithmus darf dabei nicht die asymptotische Laufzeit- schranke O(V log V + E) des Dijkstra Algorithmus überschreiten.

% Aufgabe 6.1 Stellen Sie sich vor Sie arbeiten in einer Behörde und haben ein bestimmtes Budget B ∈ N, welches Sie ausgeben müssen. Dabei ist ein Kursangebot K1, . . . , Kn gegeben wobei jeder Kurs Ki gewisse Kosten Ci ∈ N hat und einen gewissen Organisationsaufwand Ai ∈ N benötigt. Gesucht ist eine Auswahl von Kursen, so dass mindestens das Budget B ausgegeben wird und der gesamte Organisationsaufwand minimal ist.
% 1. Überlegen Sie sich einen einfachen Algorithmus, der eine Auswahl findet, die das Budget erfüllt (wenn eine existiert) und erläutern Sie kurz die Idee des Algorithmus. Der Algo- rithmus muss keine optimale Lösung finden und es muss kein Pseudocode angegeben werden.
% 2. Geben Sie sich die Rekursionsvorschrift für ein dynamisches Programm für das Problem an und eine kurze Begründung für die Korrektheit der Formel. Verwende Sie dafür eine Tabelle D, wobei D[i, b] den minimalen Organisationsaufwand beinhalten soll um mit den Kursen K1, . . . , Ki+1 genau das Budget b auszugeben.
% 3. Geben Sie einen Algorithmus in Pseudocode basierend auf der Rekursionsvorschrift an, der eine optimale Auswahl findet.
% Aufgabe 6.2 Ein Grundstück soll mit möglichst großen Einnahmen verkauft werden. Das Grund- stück ist rechteckig und liegt an einer Straße, welche durch ein Intervall (0, S) repräsentiert wird. Es liegen n Angebote mit Profiten ai für alle i ∈ {1, . . . , n} für Teile des Grundstücks vor. Diese Teile sind jeweils rechteckige Teilabschnitte mit Straßenzugang, die jeweils durch Teilintervalle (li, ri) ⊆ [0, S] repräsentiert sind. Das Ziel ist es nun eine Auswahl von Angeboten A ⊆ {1, . . . , n} zu finden, sodass die Angebote sich nicht überlappen – also (li, ri) ∩ (lj , rj ) für alle i,j ∈ A mit i ̸= j – und der Profit Pi∈A ai möglichst groß ist.
% 1. Finden Sie im gegebenen Beispiel (siehe Abbildung 1) eine optimale Auswahl.
% 2. Geben Sie die Rekursionsvorschrift für ein dynamisches Programm für das Problem an und eine kurze Begründung für die Korrektheit der Formel.
% 3. Beschreiben Sie kurz wie basierend auf der Formel ein Algorithmus für das Problem entworfen werden kann und schätzen Sie die entsprechende Laufzeit ab.
% 4. Wieviel Zeit benötigt ein Brute-Force-Algorithmus, der alle Teilmengen A ⊆ [n] überprüft?
% Hinweis: Es sei z1,...,zk eine aufsteigende Sortierung der Zahlen {0,S}∪li,ri i ∈ {1,...,n} . Ein Ansatz für die Rekursionsformel wäre es nun für alle j ∈ {1, . . . , k}, den maximalen Profit zu bestimmen, der mit Angeboten i erzielt werden kann, für die (li , ri ) ⊆ (0, zj ) gilt.
% 6 Punkte
%  6 Punkte
%    3
% 1
% 4
% 2
% 1
% 4
%   1
% 6
% 10
% 5
% 4
%        0S
% Abbildung 1: Beipiel Grundstückproblem. Die Zahlen unter den Intervallen entsprechen den
% Profiten.
% Aufgabe 6.3 Bei der Entscheidungsversion des Knapsack Problems sind Gewichte w1, . . . , wn ∈ N, Profite p1, . . . , pn ∈ N, ein Maximalgewicht W ∈ N und ein Mindestprofit P ∈ N gegeben. Es soll entschieden werden, ob eine Auswahl M ⊆ {1, . . . , n} existiert bei der beide Schranken eingehalten werden, also Pi∈M pi ≥ P und Pi∈M wi ≤ W .
% 1. Zeigen Sie KNAPSACK ∈ NP.
% 2. Zeigen Sie SUBSETSUM ≤p KNAPSACK.
% Erinnerung: Bei SUBSETSUM sind Zahlen z1, . . . , zn ∈ N und ein Zielwert T gegeben und es soll entschieden werden, ob eine Auswahl S ⊆ {1, . . . , n} existiert mit Pi∈S zi = T .
% Aufgabe 6.4 Eine NAE-k-SAT (Not-All-Equal) Formel hat folgende Form Vmi=1(z1i , . . . , zki ), wobeieineKlausel(z1i,...,zki)mitLiteralenz1i,...,zki genaudannerfülltist,wennmindestens ein Literal zu wahr und mindestens ein Literal zu falsch ausgewertet wird.
% Beispiel: Für die NAE-3-SAT Formel (x, y, z) ∧ (¬x, y, z) ist durch x = y = z = true keine erfüllende Belegung gegeben, da in diesem Fall die erste Klausel nicht erfüllt ist. Bei der Belegung x = y = true und z = false hingegen ist die Formel erfüllt.
% 1. Zeigen Sie, dass für jede erfüllende Belegung einer NAE-k-SAT, die invertierte Belegung (jede mit true belegte Variable wird mit false belegt und andersrum) ebenfalls eine erfüllende Belegung ist.
% 2. Zeigen Sie NAE-k-SAT ∈ NP.
% 3. Zeigen Sie 3-SAT ≤p NAE-4-SAT.
% Aufgabe 6.5 Bonusaufgabe. Wenden Sie auf den folgenden Graphen den Algorithmus von Floyd- Warshall an und geben Sie nach jeder Iteration der ersten Schleife die zugehörige Distanzmatrix an. Nehmen Sie an, dass Knoten zu sich selbst einen Abstand von 0 haben. Betrachten Sie die Knoten in der Reihenfolge A, B, C, D. Welches Knotenpaar besitzt den größten kürzesten Abstand?
% 6 Punkte
% 6 Punkte
% 6 Punkte

% 13 B6A
% -2 5
%      D
% -7 10
% 10
% 7C
% 5
% 3
%  -6
% 5

\exercisesheet[3]{Altklausuren}
\setcounter{section}{2019}

\end{document}