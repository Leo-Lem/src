\documentclass{exercisesheet}

\subject{Algorithmen \& Datenstrukturen}
\semester{Wintersemester 2023}
\author{Leopold Lemmermann}
% \withsolutions

\begin{document}

\createtitle

\exercisesheet[1]{Präsenzaufgaben}
\setcounter{section}{0}
\begin{eexercises}{Vollständige Induktion}{
    Beweisen Sie die folgenden Aussagen mit vollständiger Induktion:
  }
  \item $\sum_{k=1}^{n}{k} = \frac{n(n+1)}{2}\forall n \in \mathbb{Z}_{\geq 0}$.
  \item $\sum_{k=0}^{n-1} = 2^n-1\forall n \in \mathbb{Z}_{\geq 0}$.
\end{eexercises}

\begin{exercise}{Algorithmenanalyse}
  Was berechnet der Algorithmus Alg? Finden Sie einen Aufruf für den der Algorithmus besonders langsam ist. Wie verhält sich der Algorithmus bei Eingabe von rationalen oder reelen Zahlen?
  \begin{algorithm}[ht]
    \caption{Alg}
    \KwData{$a,b \in \mathbb{Z}_{>0}$}
    \KwResult{$c \in \mathbb{Z}_{>0}$}
    \If{$a=b$}{\Return{$a$}}
    \If{$a<b$}{\Return{$\text{Alg}(b-a,a)$}}
    \If{$b<a$}{\Return{$\text{Alg}(a-b,b)$}}
  \end{algorithm}
\end{exercise}

\begin{exercise}{Algorithmenentwurf}
  Entwerfen Sie einen Algorithmus, der für gegebene ganze Zahlen $a_1, \ldots, a_n \in \mathbb{Z}$ das Minimum und das Maximum bestimmt. Versuchen Sie dabei möglichst wenige Vergleiche zu verwenden ($\leq 1,5n$ sind möglich).
\end{exercise}

\newpage\setcounter{section}{1}\setcounter{subsection}{0}
\begin{eexercises}{Big-O Notation}{
    Beweisen oder widerlegen Sie die folgenden Aussagen:}
  \item $2n = \bigO(n)$
  \item $n^2 = \bigO(n)$
  \item $n \log n = \bigO(n^2)$
  \item $n^2 + n = \bigO(n^2)$
\end{eexercises}

\begin{exercise}{Algorithmus BALLSELECTION}
  Betrachten Sie den folgenden, umgangssprachlich formulierten Algorithmus. Gegeben sei eine Urne mit einer geraden Anzahl $n = 2k - 1$ weißer Kugeln und einer beliebigen Anzahl $m \geq 1$ schwarzer Kugeln.\par
  \begin{algorithm}[ht]
    \caption{BALLSELECTION}
    \While{wenigstens zwei Kugeln in der Urne sind}{
      Nimm zwei beliebige Kugeln aus der Urne heraus.\\
      \If{beide Kugeln die gleiche Farbe haben}{
        Entferne die Kugeln aus dem Spiel und lege eine neue, schwarze Kugel in die Urne.
      }
      \Else{
        Lege die weiße Kugel in die Urne zurück und entferne lediglich die schwarze Kugel aus dem Spiel.
      }
    }
  \end{algorithm}
  \noindent Beweisen Sie, dass die letzte Kugel in der Urne schwarz ist. Geben Sie dazu eine geeignete Schleifeninvariante an.
\end{exercise}

\newpage\setcounter{section}{2}\setcounter{subsection}{0}
\begin{eexercises}{Erweiterte Big-O Notation}{
    Beweisen oder widerlegen Sie die folgenden Aussagen:
  }
  \item $2n = \Omega(n)$
  \item $\log_a(n+1) = \Theta(\log_b(n+1))$ für $a,b \in \mathbb{R}_{>1}$.
  \item $n = o(2n)$
\end{eexercises}

\begin{exercise}{Mastertheorem}\setcounter{subsection}{0}
  Bestimmen Sie die Größenordnung der Funktionen, wenn möglich mittels Mastertheorem. Falls das Mastertheorem nicht anwendbar sein sollte, begründen Sie dies und verwenden stattdessen die Substitutionsmethode.
  \hint{Es ist hier kein Induktionsbeweis erforderlich bei Anwendung der Substitutionsmethode.}
  \begin{enumerate}
    \item $T_1(n) = \begin{cases}
              1                               & \text{falls } n = 1 \\
              16 \cdot T_1(\frac{n}{2}) + n^2 & \text{sonst}
            \end{cases}$
    \item $T_2(n) = \begin{cases}
              1                              & \text{falls } n = 1 \\
              8 \cdot T_2(\frac{n}{2}) + n^4 & \text{sonst}
            \end{cases}$
    \item $T_3(n) = \begin{cases}
              0                    & \text{falls } n = 0 \\
              3 \cdot T_3(n-1) + 2 & \text{sonst}
            \end{cases}$
  \end{enumerate}
\end{exercise}

\newpage\setcounter{section}{3}\setcounter{subsection}{0}
\begin{eexercises}{Sortieralgorithmen}{
    Beantworten Sie die folgenden Fragen:
  }
  \item Sortieren Sie das folgende Feld mit Radixsort. Die Basis ist 8 und die Zahlen sind zur Basis 8 angegeben.
  \begin{align*}
    142, 204, 154, 7, 104, 162, 521, 17, 262, 504, 370, 252
  \end{align*}
  \item Argumentieren Sie, dass $n$ natürliche Zahlen aus dem Bereich $\{0, 1, \ldots, n \cdot c - 1\}$ für eine Konstante $c \in \mathbb{N}$ mit Laufzeit $\bigO(n)$ sortiert werden können.
\end{eexercises}

\begin{exercise}{HEAPSORT}
  Stellen Sie das Verfahren von HEAPSORT angewendet auf die Instanz $[1,5,6,7,8,7,9,9,10]$ grafisch dar.
\end{exercise}

\newpage\setcounter{section}{4}\setcounter{subsection}{0}
\begin{exercise}{Rot-Schwarz-Bäume}
  Führen Sie die folgenden Operationen (einfügen bzw. löschen von Schlüsseln) mit einem initial leeren Rot-Schwarz-Baum aus. Geben Sie den Baum nach jedem Einfügen bzw. Löschen sowie nach jeder Rotation bzw. Umfärbung an.
  \begin{enumerate}
    \item Insert(40), Insert(48), Insert(68), Insert(55), Insert(39), Delete(40), Delete(48)
  \end{enumerate}
\end{exercise}

\begin{eexercises}{Hashtabellen}{
    Konstruieren Sie die Hashtabelle der Größe $m = 23$, die durch Einfügen der Elemente
    \begin{align*}
      47, 17, 24, 70, 22, 01, 40, 45, 36, 59
    \end{align*}
    mit der nachfolgenden Methode entsteht:
  }
  \item Divisionsmethode,
  \item Multiplikationsmethode mit $c = 1/2$ und
  \item erweiterter Divisionsmethode mit $a = 5$ und $b = 3$
\end{eexercises}

\newpage\setcounter{section}{5}\setcounter{subsection}{0}
\begin{exercise}{Topologische Sortierung}
  Finden Sie für den unten gezeigten Graph aus Abb. 1 eine Topologische Sortierung. Ist es möglich mehrere verschiedene Lösungen zu finden?
  \begin{figure}[ht]
    % TODO: add graph
  \end{figure}
\end{exercise}

\begin{exercise}{Kruskals Algorithmus}
  Bestimmen Sie mit Kruskals Algorithmus aus der Vorlesung einen minimalen Spannbaum für den Graphen aus Abb. 2. Machen Sie die Reihenfolge in der die Kanten abgearbeitet wurden deutlich.
  \begin{figure}[ht]
    % TODO: add graph
  \end{figure}
\end{exercise}

\newpage\setcounter{section}{6}\setcounter{subsection}{0}
\begin{exercise}{Independent Set}
  Ein Independent Set in einem Graphen $G = (V, E)$ ist eine Menge von Knoten $I$, so dass für je zwei Knoten $i, j \in I$ gilt, dass $\{i, j\} \notin E$. Für das Problem INDEPENDENT SET ist ein Graph $G = (V, E)$ sowie eine Zahl $k \in \mathbb{N}$ gegeben und es soll entschieden werden, ob es in $G$ ein Independent Set mit $k$ Knoten gibt.\par
  Zeigen Sie, dass das Problem INDEPENDENT SET NP-vollständig ist, indem Sie eine Reduktion von CLIQUE auf INDEPENDENT SET angeben.
\end{exercise}

\begin{exercise}{Textumbruch}
  Gegeben seien Wörter $w_1, \ldots, w_n$. Diese Wörter sollen auf Textzeilen aufgeteilt werden. Dazu seien $c(i, j)$ die Kosten, um Wörter die $w_i, \ldots, w_j$ in eine Textzeile zu schreiben für alle $1 \leq i \leq j \leq n$.\par
  Geben Sie mithilfe dynamischer Programmierung einen Algorithmus an, der Linebreaks $l_1, \ldots, l_m$ so setzt, sodass $c(1,l_1)+c(l_1 +1,l_2)+\ldots+c(l_m +1,n)$ minimiert wird.
\end{exercise}

\exercisesheet[2]{Hausaufgaben}
\setcounter{section}{1}

\begin{eexercises}{Eigenschaften der Big-O Notation}{
    Sei F die Menge der Funktionen $\mathbb{N} \rightarrow \mathbb{R}_{>0}$ und seien $f, g, h \in F$.
    Hierbei sind $\cdot, +, \max : F \times F \rightarrow F$ punktweise aufzufassen, also ist zum Beispiel $(e \cdot g)$ die durch $(e \cdot g)(n) = e(n) \cdot g(n)$ definierte Funktion.
    Beweisen oder widerlegen Sie die folgenden Aussagen:
  }
  \item Aus $e = \bigO(f)$ und $g = \bigO(h)$ folgt $(e \cdot g) = \bigO(f \cdot h)$.
  \item Aus $e = \bigO(f)$ und $g = \bigO(h)$ folgt $(e + g) = \bigO(f + h)$.
  \item Aus $e = \bigO(f)$ und $g = \bigO(h)$ folgt $(e + g) = \bigO(\max(f,h))$.
\end{eexercises}

\begin{exercise}{Größenordnung}
  Ordnen Sie die über die folgenden Ausdrücke definierten Funktionen $f_i : \mathbb{N}_{>0} \rightarrow \mathbb{R}_{\geq 0}, i = 1, \ldots, 10$ der Größe nach im Sinne der $\bigO$-Notation:
  \begin{center}
    $f_1(n) = n^3, f_2(n) = \log(\log(n)), f_3(n) = n^n, f_4(n) = \sqrt{n}, f_5(n) = n \log(n),$
    $f_6(n) = n, f_7(n) = \log_2(n), f_8(n) = 2^n, f_9(n) = n^{\frac{1}{2}}, f_{10}(n) = \log(n)$.
  \end{center}
\end{exercise}

\begin{eexercises}{Big-O Notation}{
    Beweisen oder widerlegen Sie die folgenden Aussagen:
  }
  \item $n^3 = \bigO(3n^2)$
  \item $n + \log(n) = \bigO(n \cdot \log(n))$
  \item $n^5 = \bigO(2 \cdot \log_2(n))$
  \item $3n = \bigO(2n)$
\end{eexercises}

\begin{eexercises}{Algorithmus}{
    Die folgende Funktion erhält ein Eingabearray aus natürlichen Zahlen als Eingabe.
    \begin{algorithm}[ht]
      \caption{Alg}
      \KwData{$A[i], i = 1, 2, \ldots, n, A[i] \in \mathbb{N}$}
      \KwResult{$\text{result} \in \mathbb{R}_{>0}$}
      $\text{result} \gets 0$ \\
      \For{$i = 1$ to $n$}{
        $\text{result} \gets ((i - 1) \cdot \text{result} + A[i])/i$
      }
      \Return{$\text{result}$}
    \end{algorithm}
  }
  \item Wenden Sie den Algorithmus auf ein Beispielarray der Länge mindestens drei an.
  \item Geben Sie an, was der Algorithmus berechnet.
  \item Bestimmen Sie eine geeignete Schleifeninvariante.
  \item Beweisen Sie mit Hilfe der Schleifeninvariante die Korrektheit des Algorithmus.
\end{eexercises}

\newpage\setcounter{section}{2}\setcounter{subsection}{0}
\begin{eexercises}{Erweiterte Big-O Notation}{
    Sei F die Menge der Funktionen $\mathbb{N} \rightarrow \mathbb{R}_{>0}$ und seien $f, g, h \in F$.
  }
  \item $f(n)=n^2-2n \Rightarrow f(n)=\Theta(n^2)$.
  \item $f(n)=\bigO(g(n)) \Rightarrow g(n)=\Sigma(f(n))$.
\end{eexercises}

\begin{eexercises}{Mastertheorem}{
    Bestimmen Sie die Größenordnung der Funktionen, wenn möglich mittels Mastertheorem. Falls das Mastertheorem nicht anwendbar sein sollte, begründen Sie dies und verwenden stattdessen die Substitutionsmethode.
    \hint{Es ist hier kein Induktionsbeweis erforderlich bei Anwendung der Substitutionsmethode.}
  }
  \item $T_1(n) = \begin{cases}
      1                                      & \text{falls } n = 1 \\
      4 \cdot T_1(\lfloor n/2 \rfloor) + n^2 & \text{sonst}
    \end{cases}$
  \item $T_2(n) = \begin{cases}
      1                    & \text{falls } n = 1 \\
      2 \cdot T_2(n-1) + 4 & \text{sonst}
    \end{cases}$
  \item $T_3(n) = \begin{cases}
      1                                                                                & \text{falls } n = 1 \\
      42 \cdot T_3(\lfloor n/3 \rfloor) + 39 \cdot T_3(\lceil n/3 \rceil) + 5n^3 + 69n & \text{sonst}
    \end{cases}$
\end{eexercises}

\begin{eexercises}{Rekursiver Algorithmus}{
    Gegeben sei folgender Pseudocode:
    \begin{algorithm}[ht]
      \caption{Alg($A,x,y$)}
      \KwData{$A[i], i=1,2,\ldots,n, A[i] \in \mathbb{N}, 0 < x \leq y \leq n$}
      \KwResult{$\text{result} \in \mathbb{N}_{>0}$}
      $\text{result} \gets 0$ \\
      \If{$x = y$}{
        $\text{result} \gets 1$
      }
      \ElseIf{$x < y$}{
        $k \gets x + \lfloor 2 \frac{y-x}{3} \rfloor$ \\
        $\text{value1} \gets \text{Alg}(A, x, k-1)$ \\
        $\text{value2} \gets \text{Alg}(A, k, y)$ \\
        $\text{value3} \gets 0$ \\
        $i \gets k-1, j \gets k$ \\
        \While{$A[i] = A[i-1] \land i > x$}{
          $i \gets i-1$
        }
        \While{$A[j] \leq A[j+1] \land j < y$}{
          $j \gets j+1$
        }
        \If{$A[k-1] \leq A[k]$}{
          $\text{value3} \gets j-i+1$
        }
        \Else{
          $\text{value3} \gets \max(j-k+1, k-i)$
        }
        $\text{result} \gets \max(\text{value1}, \text{value2}, \text{value3})$
      }
      \Return{$\text{result}$}
    \end{algorithm}
  }
  \item Geben Sie an, was der Algorithmus berechnet.
  \item Bestimmen Sie eine Rekurrenzgleichung, welche die Laufzeit im worst-case beschreibt. Begründen Sie die Terme in der Gleichung.
  \item Die Laufzeit von Alg liegt in $\bigO(n \log n)$. Zeigen Sie, dass Ihre Rekurrenzlgleichung diese Laufzeitschranke erfüllt.
\end{eexercises}

\begin{exercises}{Queues}
\item Erklären Sie, wie eine Queue durch zwei Stacks implementiert werden kann. Hinweis: Dazu darf Pseudocode benutzt werden, ist aber nicht zwingend erforderlich.
\item Analysieren Sie die Laufzeit der Queueoperationen aus 1. unter der Annahme, dass die Stackoperationen Zeit $\bigO(1)$ benötigen.
\end{exercises}

\newpage\setcounter{section}{3}\setcounter{subsection}{0}
\begin{eexercises}{Listen}{
    Eine zyklische einfach verkettete Liste ist eine einfach verkettete Liste, bei der zusätzlich der next-Zeiger des letzten Elements auf das erste Element der Liste zeigt. Im Folgenden seien außerdem die Elemente der Liste sortiert (aufsteigend oder absteigend, wählen sie selbst).
  }
  \item Geben Sie für die Operationen INSERT und DELETE einer solchen sortierten zyklischen Liste Pseudocode an. Achten Sie darauf, dass die Element
  \item Begründen Sie, warum ihre Implementationen aus Punkt 1 O(n) Zeit benötigen.
\end{eexercises}

\begin{exercises}{Quicksort}
\item Sortieren Sie das Array $(12, 10, 6, 3, 1, 14, 9)$ mithilfe des QUICKSORT-Algorithmus aus der Vorlesung. Wählen Sie hierzu das Pivotelement wie im QUICKSORT-Algorithmus aus der Vorlesung vorgegeben. Stellen Sie den Inhalt des Arrays nach jedem Aufruf von PARTITION dar.
\item Welchen Einfluss auf die Laufzeit hat allgemein die Auswahl des Pivotelements bei QUICKSORT? Skizzieren Sie worst-case- und best-case-Eingaben für eine konkrete Auswahl des Pivot- elements (z.B. immer am linken oder rechten Rand). Begründen Sie Ihre Antwort.
\item Ist der QUICKSORT-Algorithmus aus der Vorlesung stabil? Begründen Sie Ihre Antwort.
\end{exercises}

\begin{exercise}{Heaps}
  Zeigen Sie per struktureller Induktion, dass in jedem (binären) Heap $H$ für die Anzahl Blätter $b(H)$ und Anzahl Nicht-Blätter $n(H)$ gilt:
  \begin{equation*}
    n(H) \leq b(H) \leq n(H)+1
  \end{equation*}
\end{exercise}

\begin{eexercises}{SELECT-Algorithmus}{
    In der Vorlesung wurde der SELECT-Algorithmus besprochen, der das Eingabearray in 5er-Gruppen aufteilt. Der Algorithmus kann aber auch leicht für andere Gruppengrößen angepasst werden. Sie können (wie in der Vorlesung) davon ausgehen, dass die Element
  }
  \item Untersuchen Sie ob die Argumente für die Laufzeitabschätzung auch für 7er-Gruppen funktionieren. Finden und begründen Sie dafür eine entsprechende Rekurrenzgleichung und schätzen Sie die Laufzeit ab.
  \item Finden und begründen Sie eine entsprechende Rekurrenzgleichung für den Fall mit 3er-Gruppen. Geben Sie zudem eine kurze Einschätzung dazu an, ob in diesem Fall auch eine lineare Laufzeit erreicht wird (hier ist kein formaler Beweis gefordert).
  \item Stellen Sie eine Rekurrenzgleichung für eine Variante von QUICKSORT auf, die den SELECT-Algorithmus für die Wahl des Pivotelements verwendet und schätzen Sie die Laufzeit mit dem Mastertheorem ab.
\end{eexercises}

\newpage\setcounter{section}{4}\setcounter{subsection}{0}
\begin{eexercises}{Suchbäume}{
    Im Folgenden betrachten wir Zahlen $n = 2^k - 1$ mit $k \in \mathbb{N}_{>0}$.
  }
  \item In welcher Reihenfolge sollten die Zahlen $1, 2, \ldots, n$ in einen initial leeren Suchbaum eingefügt werden, damit dieser möglichst balanciert/niedrig ist? Begründen Sie Ihre Antwort kurz.
  \item Geben Sie Pseudocode für ein Verfahren mit $\bigO(n)$ Laufzeit an, dass einen entsprechenden Suchbaum erzeugt. Geben Sie eine kurze Begründung an, weshalb ihr Verfahren korrekt ist und die entsprechende Laufzeit erreicht.
  \item Entwerfen Sie einen Algorithmus, der in linearer Laufzeit einen beliebigen Suchbaum mit $n$ Einträgen in einen balancierten Suchbaum umwandelt. Begründen Sie kurz Laufzeit und Korrektheit des Verfahrens. Pseudocode muss nicht angegeben werden.
\end{eexercises}

\begin{eexercises}{Hashing}{
    Wir betrachten Hashing mit offener Adressierung und quadratischer Sondierung, also eine Hashfunktion der Form $h(k, i) = (h_0(k) + c_1i + c_2i^2) \mod m$. Es seien $h_0(x) = x \mod 13$, sowie $c_1 = c_2 = 12$.
  }
  \item Fügen Sie die Elemente $938, 1243, 10026, 71, 831, 555, 142, 768, 301, 176, 9347, 32418$ und $360$ in der angegebenen Reihenfolge in eine Hashtabelle der Größe $m = 13$ ein. Stellen Sie das Verfahren graphisch dar. Dabei sollte klar werden wo Kollisionen auftreten und wie diese aufgelöst werden.
  \item Was passiert, wenn als letzte Zahl anstatt der $360$ eine $359$ eingefügt wird?
\end{eexercises}

\begin{eexercises}{Graphen}{
    Es sei $G = (V, E)$ ein ungerichteter Graph. Zeigen Sie:
  }
  \item Es gilt $\sum_{v \in V} \text{deg}(v) = 2|E|$.
  \item Wenn $G$ zusammenhängend (verbunden) ist, dann gilt $|E| \geq |V| - 1$.
  \item Wenn $G$ azyklisch ist, dann gilt $|E| \leq |V| - 1$.
  \item Wenn $G$ ein Baum ist, dann gilt $|E| = |V| - 1$.
  \item Bäume sind kantenmaximal kreisfrei und kantenminimal zusammenhängend. Wird also eine Kante hinzugefügt bzw. entfernt, so ist der entsprechende Graph nicht mehr kreisfrei bzw. zusammenhängend.
\end{eexercises}

\begin{eexercises}{Rot-Schwarz-Bäume}{
    Gegeben seien zwei Rot-Schwarz-Bäume $B_1$ und $B_2$ und ein Element $x \in \mathbb{Z}$, sodass für alle $x_1 \in B_1$ und $x_2 \in B_2$ gilt: $x_1.\text{key} \leq x \leq x_2.\text{key}$.
  }
  \item Beschreiben Sie einen Algorithmus, welcher aus der Vereinigung $B = B_1 \cup \{x\} \cup B_2$ einen neuen Rot-Schwarz-Baum in $\bigO(\log n)$ Zeit berechnet, wobei $n$ die Gesamtanzahl aller Knoten in $B_1$ und $B_2$ ist.
  \item Begründen Sie, warum eurer Algorithmus die Laufzeitschranke einhält.
\end{eexercises}

\newpage\setcounter{section}{5}\setcounter{subsection}{0}
\begin{eexercises}{Graphen}{
    Betrachten Sie obigen Graphen $G$ aus Abb. 1. Wenden Sie jeweils das verlangte Verfahren an bzw. beantworten Sie die Frage oder begründen Sie, warum dies nicht geht.
    % TODO: add graph
  }
  \item Ermitteln Sie mit der Breitensuche einen Breitensuchbaum (BFS-Baum). Starten Sie den Algorithmus bei $s$.
  \item Ist das Ergebnis der Breitensuche eindeutig?
  \item Ermitteln Sie mit der Tiefensuche einen Tiefensuchwald und insb. die Zeiten $u.d$ und $u.f$ für jeden Knoten. Starten Sie den Algorithmus wieder bei $s$.
  \item Geben Sie eine topologische Sortierung des Graphen $G$ an.
  \item Bestimmen Sie mit dem Algorithmus aus der Vorlesung die starken Zusammenhangskomponenten von $G$.
\end{eexercises}

\begin{eexercises}{Minimale Spannbäume}{
    % TODO: add graph
  }
  \item Bestimmen Sie mit Hilfe des Algorithmus von Prim einen minimalen Spannbaum des Graphen und skizzieren diesen. Der Startknoten ist grau gefärbt. Geben Sie zusätzlich die Reihenfolge an, in der die Kanten des Spannbaums gemäß Algorithmus hinzugefügt werden.
  \item Sei $G = (V, E)$ ein zusammenhängender gewichteter Graph mit Gewichtsfunktion $w : E \rightarrow \mathbb{R}$ und $T = (V, E_0)$ ein minimaler Spannbaum von $G$.
  \begin{enumerate}
    \item Für eine beliebige Kante $e \in E \setminus E_0$ wird deren Gewicht $w(e)$ erhöht. Wir bezeichnen diesen modifizierten Graphen als $G'$. Zeigen Sie, dass $T$ ebenfalls ein minimaler Spannbaum von $G'$ ist.
    \item Für eine beliebige Kante $e \in E$ wird deren Gewicht $w(e)$ verringert. Wir bezeichnen diesen modifizierten Graphen als $G'$. Konstruieren Sie einen Algorithmus, welcher einen minimalen Spannbaum von $G'$ mit Hilfe von $T$ in Zeit $\bigO(|V| + |E|)$ berechnet. Hinweis: Hier ist kein Pseudocode gefordert.
  \end{enumerate}
\end{eexercises}

\begin{eexercises}{Brückenkanten}{
    % TODO: add graph
    Eine Kante $e \in E$ eines zusammenhängenden, ungerichteten Graphen $G = (V, E)$ heißt Brückenkante, falls der Graph $G_0 = (V, E \setminus \{e\})$, der durch das Entfernen der Kante $e$ aus $G$ entsteht, nicht mehr zusammenhängend ist.
  }
  \item Bestimmen Sie alle Brückenkanten im oben abgebildeten Graphen.
  \item Geben Sie einen Algorithmus in Pseudocode an, der in Laufzeit $\bigO(|V| \cdot |E| + |E|^2)$ entscheidet, ob ein zusammenhängender, ungerichteter Graph $G$ eine Brückenkante besitzt oder nicht.
  \item Zeigen Sie, dass Ihr Algorithmus korrekt arbeitet.
  \item Zeigen Sie, dass Ihr Algorithmus die Laufzeitschranke einhält.
\end{eexercises}

\begin{eexercises}{Dijkstras Algorithmus}{
    % TODO: add graph
  }
  \item Wenden Sie Dijkstras Algorithmus auf den Graphen aus Abb. 4 an. Beginnen Sie bei Knoten $s$. Geben Sie tabellarisch die am Ende jeder Iteration der While-Schleife in der Queue enthaltenen Keys an.
  \item Modifizieren Sie den Pseudocode von Dijkstras Algorithmus so, dass er als Input einen Graphen, einen Startknoten $s$ und einen Zielknoten $v$ als Input nimmt und als Output die explizite Darstellung des kürzesten Pfades von $s$ nach $v$ als Sequenz von Knoten ausgibt. Begründen Sie kurz die Korrektheit Ihrer Lösung. Die Laufzeit des modifizierten Algorithmus darf dabei nicht die asymptotische Laufzeitschranke $\bigO(|V| \log |V| + |E|)$ des Dijkstra Algorithmus überschreiten.
\end{eexercises}

\newpage\setcounter{section}{6}\setcounter{subsection}{0}
\begin{eexercises}{Kursauswahl}{
    Stellen Sie sich vor Sie arbeiten in einer Behörde und haben ein bestimmtes Budget $B \in \mathbb{N}$, welches Sie ausgeben müssen. Dabei ist ein Kursangebot $K_1, \ldots, K_n$ gegeben wobei jeder Kurs $K_i$ gewisse Kosten $C_i \in \mathbb{N}$ hat und einen gewissen Organisationsaufwand $A_i \in \mathbb{N}$ benötigt. Gesucht ist eine Auswahl von Kursen, so dass mindestens das Budget $B$ ausgegeben wird und der gesamte Organisationsaufwand minimal ist.
  }
  \item Überlegen Sie sich einen einfachen Algorithmus, der eine Auswahl findet, die das Budget erfüllt (wenn eine existiert) und erläutern Sie kurz die Idee des Algorithmus. Der Algorithmus muss keine optimale Lösung finden und es muss kein Pseudocode angegeben werden.
  \item Geben Sie sich die Rekursionsvorschrift für ein dynamisches Programm für das Problem an und eine kurze Begründung für die Korrektheit der Formel. Verwende Sie dafür eine Tabelle $D$, wobei $D[i, b]$ den minimalen Organisationsaufwand beinhalten soll um mit den Kursen $K_1, \ldots, K_i+1$ genau das Budget $b$ auszugeben.
  \item Geben Sie einen Algorithmus in Pseudocode basierend auf der Rekursionsvorschrift an, der eine optimale Auswahl findet.
\end{eexercises}

\begin{eexercises}{Grundstückproblem}{
    Ein Grundstück soll mit möglichst großen Einnahmen verkauft werden. Das Grundstück ist rechteckig und liegt an einer Straße, welche durch ein Intervall $(0, S)$ repräsentiert wird. Es liegen $n$ Angebote mit Profiten $a_i$ für alle $i \in \{1, \ldots, n\}$ für Teile des Grundstücks vor. Diese Teile sind jeweils rechteckige Teilabschnitte mit Straßenzugang, die jeweils durch Teilintervalle $(l_i, r_i) \subseteq [0, S]$ repräsentiert sind. Das Ziel ist es nun eine Auswahl von Angeboten $A \subseteq \{1, \ldots, n\}$ zu finden, sodass die Angebote sich nicht überlappen – also $(l_i, r_i) \cap (l_j, r_j)$ für alle $i,j \in A$ mit $i \neq j$ – und der Profit $\sum_{i \in A} a_i$ möglichst groß ist.
    \hint{Es sei $z_1, \ldots, z_k$ eine aufsteigende Sortierung der Zahlen $\{0, S\} \cup \bigcup_{i=1}^n \{l_i, r_i\}$. Ein Ansatz für die Rekursionsformel wäre es nun für alle $j \in \{1, \ldots, k\}$, den maximalen Profit zu bestimmen, der mit Angeboten $i$ erzielt werden kann, für die $(l_i, r_i) \subseteq (0, z_j)$ gilt.}
  }
  \item Finden Sie im gegebenen Beispiel (siehe Abbildung 1) eine optimale Auswahl.
  \item Geben Sie die Rekursionsvorschrift für ein dynamisches Programm für das Problem an und eine kurze Begründung für die Korrektheit der Formel.
  \item Beschreiben Sie kurz wie basierend auf der Formel ein Algorithmus für das Problem entworfen werden kann und schätzen Sie die entsprechende Laufzeit ab.
  \item Wieviel Zeit benötigt ein Brute-Force-Algorithmus, der alle Teilmengen $A \subseteq [n]$ überprüft?
\end{eexercises}

\begin{eexercises}{Entscheidungsproblem des Knapsack Problems}{
    Bei der Entscheidungsversion des Knapsack Problems sind Gewichte $w_1, \ldots, w_n \in \mathbb{N}$, Profite $p_1, \ldots, p_n \in \mathbb{N}$, ein Maximalgewicht $W \in \mathbb{N}$ und ein Mindestprofit $P \in \mathbb{N}$ gegeben. Es soll entschieden werden, ob eine Auswahl $M \subseteq \{1, \ldots, n\}$ existiert bei der beide Schranken eingehalten werden, also $\sum_{i \in M} p_i \geq P$ und $\sum_{i \in M} w_i \leq W$.
  }
  \item Zeigen Sie $\text{KNAPSACK} \in \text{NP}$.
  \item Zeigen Sie $\text{SUBSETSUM} \leq_p \text{KNAPSACK}$.
  \hint{Bei SUBSETSUM sind Zahlen $z_1, \ldots, z_n \in \mathbb{N}$ und ein Zielwert $T$ gegeben und es soll entschieden werden, ob eine Auswahl $S \subseteq \{1, \ldots, n\}$ existiert mit $\sum_{i \in S} z_i = T$.}
\end{eexercises}

\begin{eexercises}{NAE-k-SAT}{
    Eine NAE-k-SAT (Not-All-Equal) Formel hat folgende Form $\bigvee_{m=1}^k (z_{1m} \land \ldots \land z_{km})$, wobei eine Klausel $(z_{1m}, \ldots, z_{km})$ mit Literalen $z_{1m}, \ldots, z_{km}$ genau dann erfüllt ist, wenn nicht alle Literale gleichzeitig wahr sind.
  }
  \item Zeigen Sie, dass NAE-k-SAT NP-vollständig ist.
  \item Zeigen Sie, dass NAE-3-SAT NP-vollständig ist.
  \item Zeigen Sie, dass NAE-3-SAT NP-vollständig ist.
\end{eexercises}

\begin{exercise}{Floyd-Warshall Algorithmus}
  Wenden Sie auf den folgenden Graphen den Algorithmus von Floyd-Warshall an und geben Sie nach jeder Iteration der ersten Schleife die zugehörige Distanzmatrix an. Nehmen Sie an, dass Knoten zu sich selbst einen Abstand von 0 haben. Betrachten Sie die Knoten in der Reihenfolge A, B, C, D. Welches Knotenpaar besitzt den größten/kürzesten Abstand?
  % TODO: graph
\end{exercise}

\exercisesheet[3]{Altklausuren}
\setcounter{section}{2019}



\end{document}