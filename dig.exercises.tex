\documentclass{exercisesheet}

\title{Aufgaben für Datenschutz in der Informationsgesellschaft}
\author{Leopold Lemmermann}
% \solutions

\begin{document}

\createtitle

\exercisesheet{SVS Fragenkatalog}
  \begin{exercise}{Geschichte des Datenschutzes}
      \item Was hat das Bundesverfassungsgericht 1983 an der geplanten Volkszählung zu beanstanden?
      \item Beschreiben Sie zwei Beispiele, wie durch eine unkontrollierte Bekanntgabe \& Verarbeitung von persönlichen Daten die Freiheit eines Menschen gehemmt sein kann.
  \end{exercise}

  \begin{exercise*}{Anwendungsbereich der DSGVO}
    Ein in Hamburg lebender US-Amerikaner bestellt bei einen chinesischen Online-Versandhändler einen Elektrobausatz. Bei der Bestellung gibt er Name, Anschrift \& seine E-Mail-Adresse an. Prüfen Sie stichpunktartig, ob auf diese Bestellung die DSGVO anwendbar wäre.
  \end{exercise*}

  \begin{exercise*}{Datenschutz-Grundverordnung}
    Welche Auswirkungen hat die Datenschutz-Grundverordnung (DSGVO) auf das europäische Datenschutzrecht \& die bisherigen naeonalen Regelungen?
  \end{exercise*}

  \begin{exercise*}{Grundlagen der Datenverarbeitung}
    Das Hotel H erfasst von seinem Kunden K während dessen Aufenthalt alle Bestellungen \& Wünsche \& speichert diese für die Nutzung bei einem eventuell erneuten Aufenthalt von K.
    \begin{enumerate}
      \item Auf welcher Grundlage könnte dies rechtmäßig erfolgen? Nennen Sie zwei Grundlagen jeweils mit Fundstelle (Paragraph, Arekel etc.) nach BDSG oder DSGVO. Keine Begründung erforderlich.
      \item Das Hotel H fragt Sie um Rat, mit welcher Grundlage es in der Praxis "auf der sicheren Seite" wäre. Begründen Sie Ihre Wahl.
      \item Nennen Sie je ein legiemes Interesse, das das Hotel H bzw. der Kunde K an der Verarbeitung bzw. deren Ausschluss haben könnte.
    \end{enumerate}
  \end{exercise*}



% raw content here
% Exercise 1.1 Geschichte des Datenschutzes
% a) Was haYe das Bundesverfassungsgericht 1983 an der geplanten Volkszählung zu beanstanden?
% b) BiYe beschreiben Sie zwei Beispiele, wie durch eine unkontrollierte Bekanntgabe & Verarbeitung von persönlichen Daten die Freiheit eines Menschen gehemmt sein kann.
% Exercise 1.2 Anwendungsbereich der DSGVO
% Ein in Hamburg lebender US-Amerikaner bestellt bei einen chinesischen Online-Versandhändler einen Elektrobausatz. Bei der Bestellung gibt er Name, Anschric & seine E-Mail-Adresse an. Prüfen Sie sechpunktareg, ob auf diese Bestellung die DSGVO anwendbar wäre.
% Exercise 1.3 Datenschutz-Grundverordnung
% Welche Auswirkungen hat die Datenschutz-Grundverordnung (DSGVO) auf das europäische Datenschutzrecht & die bisherigen naeonalen Regelungen?
% Exercise 1.4 Grundlagen der Datenverarbeitung
% Das Hotel H erfasst von seinem Kunden K während dessen Aufenthalt alle Bestellungen & Wünsche & speichert diese für die Nutzung bei einem eventuell erneuten Aufenthalt von K.
% a) Auf welcher Grundlage könnte dies rechtmäßig erfolgen? Nennen Sie zwei Grundlagen jeweils mit Fundstelle (Paragraph, Arekel etc.) nach BDSG oder DSGVO. Keine Begründung erforderlich.
% b) Das Hotel H fragt Sie um Rat, mit welcher Grundlage es in der Praxis "auf der sicheren Seite" wäre. Begründen Sie Ihre Wahl.
% c) Nennen Sie je ein legiemes Interesse, das das Hotel H bzw. der Kunde K an der Verarbeitung bzw. deren Ausschluss haben könnte.
% Exercise 1.5 Betrieblicher DatenschutzbeauDragter
% Für betriebliche Datenschutzbeaucragte werden Fachkunde & Zuverlässigkeit gefordert. a) Was versteht man unter Fachkunde des betrieblichen Datenschutzbeaucragten?
% b) Was versteht man unter Zuverlässigkeit des betrieblichen Datenschutzbeaucragten?
% Exercise 1.6 Anonymisierung & Pseudonymisierung
% a) Was bedeuten die Begriffe Anonymisierung & Pseudonymisierung von personenbezogenen Daten?
% b) Warum findet sich in Art. 4 DSGVO nur der Begriff "Pseudonymisierung", nicht jedoch "Anonymisierung"?
% c) BiYe beschreiben Sie am Beispiel von IP-Adressen in Logdateien, wie die Anonymisierung & Pseudonymisierung von personenbezogenen Daten geschehen könnte.
% Exercise 1.7 FakMsche Anonymität
% In der Rechtsdogmaek wird zwischen fakescher & absoluter Anonymität unterschieden.
% a) Auf welche Textstelle im BDSG bezieht sich diese Unterscheidung & wie grenzen sich die Begriffe ab?
% b) Nennen Sie ein Beispiel, wie durch technischen FortschriY ein fakesch anonymisiertes Datum wieder personenbezogen werden kann.
% Exercise 1.8 Anonymisierte Daten
% Sind anonymisierte Daten personenbezogene Daten? BiYe begründen Sie Ihre Antwort!
%           University of Hamburg Leopold Lemmermann
%   DIG SVS Fragenkatalog Dr. Hannes Federath
% 3
% Exercise 1.9 Digitalisierung historischer Telefonbücher
% Ein Berliner Unternehmen möchte 100 Jahre alte, gedruckte Telefonbücher digitalisieren & im Internet veröffentlichen. Ist dies datenschutzrechtlich zulässig?
% Hinweis: Bi*e lassen Sie bei Ihrer Antwort urheber- & lizenzrechtliche Aspekte außer Acht.
% Exercise 1.10 Recht auf AuskunD
% Bei einem Hoseng-Provider geht per Post eine Beschwerde über die unberechegte Speicherung von Daten durch einen Online-Shop ein. Der Beschwerdeführer möchte vom Hoseng-Provider des Online-Shops wissen, welche Daten über ihn von dem dort gehosteten Online-Shop zu seiner Person gespeichert werden. BiYe formulieren Sie ein Antwortschreiben des Hoseng-Providers.
% Exercise 1.11 Löschung
% Der Hamburger Verein Tier-Interessen & ehrenamtliche ReYung von Tieren (kurz TIeR) hat ein Problem mit einem Hundehasser. Dieser hat aus Hass auf den Verein ein Protokoll der letzten Vorstandssitzung auf Facebook veröffentlicht. Kann TIeR von Facebook die Löschung der Daten nach Art. 17 DSGVO verlangen? Begründen Sie Ihre Antwort.
% Exercise 1.12 Löschung II
% Stellt das Löschen von personenbezogenen Daten eine Verarbeitung dar? Begründen Sie anhand der DSGVO.
% Exercise 1.13 Datenminimierung
% Bewerten Sie folgende Aussage zur DSGVO: "Der Grundsatz der Datenminimierung verbietet Verarbeitungstäegkeiten, bei denen personenbezogene Daten in großem Umfang erforderlich sind." Ist diese Aussage korrekt? Begründen Sie anhand der DSGVO.
% Exercise 1.14 Betroffenenrechte
% Herrn K wird neuerdings verwehrt, im Onlineshop S auf Rechnung zu bezahlen. Auf Nachfrage nennt S eine erst nach wiederholter Mahnung bezahlte Rechnung aus einer früheren Bestellung als Ursache. Die deshalb schlechte Bonität führe zu einem automaeschen Ausschluss von der Zahlung per Rechnung.
% a) Herr K bestreitet die verspätete Zahlung & fordert S auf, die falschen Angaben ab sofort nicht mehr zu nutzen & zu korrigieren. Hat K nach der DSGVO das Recht zu diesen Forderungen? Begründen Sie Ihre Antwort.
% b) Herr K hat nach dem Ärger mit S nun doch genug & verlangt von S die Löschung seiner personenbezogenen Daten. AnstaY zu löschen, anonymisiert S die Daten, um sie noch für Staeseken nutzen zu können. Ist eine Anonymisierung anstelle einer Löschung zulässig? Begründen Sie.
% c) Herr K findet außerdem, dass das Shopsystem von S durch den automaeschen Ausschluss der Bezahlopeon Rechnung gegen den Art. 22 DSGVO verstoßen hat & möchte sich bei der Aufsichtsbehörde beschweren. Hat seine Beschwerde Aussicht auf Erfolg? Begründen Sie.
% Exercise 1.15 Rechtmäßigkeit der Verarbeitung
% a) In welchen Fällen erlaubt die DSGVO die Verarbeitung von personenbezogenen Daten auch ohne die Einwilligung des Betroffenen? Nennen Sie drei Fälle samt Fundstelle im Gesetz & beschreiben Sie diese kurz.
% b) Um ihrer Schwester eine Freude zu machen, beaucragt Frau K den Blumenhändler B der Schwester am Geburtstag Blumen zu bringen. Dazu teilt K dem Blumenhändler B die Anschric & das Geburtsdatum der Schwester mit. Prüfen Sie die Datenverarbeitung durch K & B hinsichtlich ihrer Rechtmäßigkeit nach der DSGVO. Sechpunkte genügen.
% Exercise 1.16 Zweckbindung
% Erklären Sie kurz den Begriff Zweckbindung. Wo wird die Zweckbindung in der DSGVO geregelt & welche Ausnahmen gibt es?
% Exercise 1.17 Einwilligung
%          University of Hamburg Leopold Lemmermann

%   DIG SVS Fragenkatalog Dr. Hannes Federath
% a) Weshalb kann die Einwilligung als Rechtsgrundlage für die Verarbeitung von Daten eines Angestellten
% (Betroffener) durch dessen Arbeitgeber (Verantwortlicher) als problemaesch angesehen werden?
% b) Was wird in Zusammenhang mit einer Einwilligung als Kopplungsverbot bezeichnet?
% c) In einem Online-Bestellformular soll ein Kästchen (Checkbox) zur Einwilligung in die Verarbeitung zu Werbezwecken eingebaut werden. Begründen Sie anhand der DSGVO, ob das Kästchen standardmäßig angekreuzt oder leer sein sollte, damit die Einwilligung wirksam ist. Nennen Sie die Fundstelle im Gesetz.
% Exercise 1.18 Verarbeitung durch Dienstleister
% Das Start-up-Unternehmen S geht mit einem neuen, innovaeven Telemediendienst auf den Markt. Da es noch über keine eigene Infrastruktur verfügt, beaucragt es Hoseng-Dienstleister H mit dem Hoseng des Dienstes. Die Administraeon des Dienstes erfolgt durch Mitarbeiter des Unternehmens S über Fernzugriff. Sowohl S als auch H sitzen in Deutschland.
% a) Liegt hier eine Aucragsdatenverarbeitung oder Funkeonsübertragung vor? Begründen Sie. Gehen Sie davon aus, dass personenbezogene Daten verarbeitet werden.
% b) In welchem der beiden Fälle Aucragsdatenverarbeitung oder Funkeonsübertragung läge eine ÜbermiYlung im Sinne des BDSG vor? Begründen Sie.
% Exercise 1.19 Data ProtecMon by Design
% Art. 25 (1) DSGVO benennt Aspekte, die ein Verantwortlicher bei der Auswahl geeigneter technischer & organisatorischer Maßnahmen zu berücksichegen hat. Beschreiben Sie den Einfluss von zwei der genannten Aspekte auf die Eignung einer Maßnahme.
% Exercise 1.20 Safe Harbour
% Der Europäische Gerichtshof (EuGH) hat im Oktober 2015 die Entscheidung der EU-Kommission zum "Sicheren Hafen" (Safe Harbour) anulliert.
% a) Was versteht man unter Safe Harbour?
% b) Warum hat der EuGH die Safe Harbour-Entscheidung der EU-Kommission anulliert? c) Was ändert sich mit Privacy Shield?
% Exercise 1.21 Datenschutzerklärung
% Sie haben folgende Datenschutzerklärung einer Firma vorliegen. Die Firma bietet eine Reihe von Apps für Handys an. Bei einer Überprüfung der Datenschutzerklärung fallen Ihnen Punkte auf, die unter Berufung auf die DSGVO anfechtbar sind.
% BiYe nennen Sie fünf Punkte, die Ihrer Meinung nach nicht erfüllt sind oder gegen die verstoßen wird, jeweils mit Benennung des Arekels & Erläuterung warum & gegen was verstoßen wird oder was nicht erfüllt wird.
% Exercise 1.22 Privacy Design Strategies
% BiYe nennen Sie je ein Beispiel für die folgenden Privacy Design Strategies im Kontext von Big Data: Demonstrate, Separate, Enforce, Inform, Hide, Minimise, Perturbate, Aggregate, Control.
% Hinweis: BiYe geben Sie je eine Antwort für diese neun Privacy Design Strategies.

\end{document}