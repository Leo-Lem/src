\documentclass{exercisesheet}

\subject{Datenschutz in der Informationsgesellschaft}
\semester{Wintersemester 2023}
\author{Leopold Lemmermann}
% \withsolutions

\begin{document}

\createtitle

\exercisesheet{SVS Fragenkatalog}
\begin{exercises}{Geschichte des Datenschutzes}
\item Was hat das Bundesverfassungsgericht 1983 an der geplanten Volkszählung zu beanstanden?
\item Beschreiben Sie zwei Beispiele, wie durch eine unkontrollierte Bekanntgabe \& Verarbeitung von persönlichen Daten die Freiheit eines Menschen gehemmt sein kann.
\end{exercises}

\begin{solutions}
  \item Zu viele (teils nicht zweckmäßige) Daten erhoben, Datensicherheit nicht gewährleistet.
  \item zB. Erstellung von Persönlichkeitsprofilen, Verweigerung von Krediten, Versicherungen, Arbeitsplätzen, etc.
\end{solutions}

\begin{exercise}{Anwendungsbereich der DSGVO}
  Ein in Hamburg lebender US-Amerikaner bestellt bei einen chinesischen Online-Versandhändler einen Elektrobausatz. Bei der Bestellung gibt er Name, Anschrift \& seine E-Mail-Adresse an. Prüfen Sie stichpunktartig, ob auf diese Bestellung die DSGVO anwendbar wäre.

  \begin{solution}
    \begin{itemize}
      \item Name, Anschrift \& E-Mail-Adresse sind personenbezogene Daten. \checkmark
      \item Der US-Amerikaner ist eine natürliche Person. \checkmark
      \item Der Versandhändler ist ein Unternehmen. \checkmark
      \item Der Versandhändler hat eine Niederlassung in der EU. \xmark
      \item Der Versandhändler bietet Waren \& Dienstleistungen in der EU an. \checkmark
      \item Die Bestellung ist eine Verarbeitung personenbezogener Daten. \checkmark
    \end{itemize}
  \end{solution}
\end{exercise}

\begin{exercise}{Datenschutz-Grundverordnung}
  Welche Auswirkungen hat die Datenschutz-Grundverordnung (DSGVO) auf das europäische Datenschutzrecht \& die bisherigen nationalen Regelungen?

  \begin{solution}
    Die DSGVO ist unmittelbar geltendes Recht in allen Mitgliedsstaaten der EU. Sie ersetzt die bisherigen nationalen Regelungen, soweit diese nicht mit der DSGVO vereinbar sind.
  \end{solution}
\end{exercise}

\begin{eexercises}{Grundlagen der Datenverarbeitung}{
    Das Hotel H erfasst von seinem Kunden K während dessen Aufenthalt alle Bestellungen \& Wünsche und speichert diese für die Nutzung bei einem eventuell erneuten Aufenthalt von K.
  }
  \item Auf welcher Grundlage könnte dies rechtmäßig erfolgen? Nennen Sie zwei Grundlagen jeweils mit Fundstelle (Paragraph, Artikel etc.) nach BDSG oder DSGVO. Keine Begründung erforderlich.
  \item Das Hotel H fragt Sie um Rat, mit welcher Grundlage es in der Praxis "auf der sicheren Seite" wäre. Begründen Sie Ihre Wahl.
  \item Nennen Sie je ein legitimes Interesse, das das Hotel H bzw der Kunde K an der Verarbeitung bzw deren Ausschluss haben könnte.
\end{eexercises}

\begin{solutions}
  \item Art. 6 Abs. 1 lit. a) DSGVO (Einwilligung) \& Art. 6 Abs. 1 lit. f) DSGVO (berechtigtes Interesse)
  \item Art. 6 Abs. 1 lit. b) DSGVO (Einwiligung), da wenig Auslegungsbedarf \& keine Abwägung nötig.
  \item
  Legitimes Interesse an Verarbeitung H: zB. Wiedererkennung des Kunden, Verbesserung des Kundenservice \& K: zB. Wiedererkennung des Hotels, Verbesserung des Kundenservice\par
  Legitimes Interesse an Ausschluss H: zB. Vermeidung von Aufwand, Vermeidung von Verantwortung \& K: zB. Vermeidung von Werbung, Vermeidung von Datenverlust
\end{solutions}

\begin{eexercises}{Betrieblicher Datenschutzbeauftragter}{
    Für betriebliche Datenschutzbeauftragte werden Fachkunde \& Zuverlässigkeit gefordert.
  }
  \item Was versteht man unter Fachkunde des betrieblichen Datenschutzbeauftragten?
  \item Was versteht man unter Zuverlässigkeit des betrieblichen Datenschutzbeauftragten?
\end{eexercises}

\begin{solutions}
  \item Fachkunde des betrieblichen Datenschutzbeauftragten: Kenntnisse des Datenschutzrechts \& der Technik der Datenverarbeitung.
  \item Zuverlässigkeit des betrieblichen Datenschutzbeauftragten: Keine Straftaten, keine Datenschutzverstöße, keine Interessenkonflikte.
\end{solutions}

\begin{exercises}{Anonymisierung \& Pseudonymisierung}
\item Was bedeuten die Begriffe Anonymisierung \& Pseudonymisierung von personenbezogenen Daten?
\item Warum findet sich in Art. 4 DSGVO nur der Begriff "Pseudonymisierung", nicht jedoch "Anonymisierung"?
\item Beschreiben Sie am Beispiel von IP-Adressen in Logdateien, wie die Anonymisierung \& Pseudonymisierung von personenbezogenen Daten geschehen könnte.
\end{exercises}

\begin{solutions}
  \item Anonymisierung: Person kann nicht mehr identifiziert werden. Pseudonymisierung: Person kann nur mit zusätzlichen Informationen identifiziert werden.
  \item Anonyme Daten werden von der DSGVO grundsätzlich nicht erfasst. Pseudonyme Daten sind allerdings als personenbezogen definiert, da sie mit zusätzlichen Informationen einer Person zugeordnet werden können. Daher ist die Pseudonymisierung explizit erwähnt.
  \item Anonymisierung: Alle IP-Adressen werden durch eine IP-Adresse ersetzt, die nicht einer Person zugeordnet werden kann. Pseudonymisierung: Alle IP-Adressen werden durch eine IP-Adresse ersetzt, die nur mit zusätzlichen Informationen einer Person zugeordnet werden kann.
\end{solutions}

\begin{eexercises}{Faktische Anonymität}{
    In der Rechtsdogmatik wird zwischen faktischer \& absoluter Anonymität unterschieden.
  }
  \item Auf welche Textstelle im BDSG bezieht sich diese Unterscheidung \& wie grenzen sich die Begriffe ab?
  \item Nennen Sie ein Beispiel, wie durch technischen Fortschritt eine faktisch anonymisiertes Datum wieder personenbezogen werden kann.
\end{eexercises}

\begin{solutions}
  \item §3 Abs. 6 BDSG (inzwischen aufgehoben): Faktisch anonymisierte Daten sind Daten, die nicht mehr einer Person zugeordnet werden können. Absolut anonymisierte Daten sind Daten, die nie einer Person zugeordnet werden können.
  \item Moore's law $\to$ Rechenleistung erhöht sich startk $\to$ Brute-Force-Angriffe auf Hashes werden uU. möglich.
\end{solutions}


\begin{exercise}{Anonymisierte Daten}
  Sind anonymisierte Daten personenbezogene Daten? Begründen Sie Ihre Antwort!

  \begin{solution}
    Nein, anonymisierte Daten sind keine personenbezogenen Daten, da sie nicht mehr einer Person zugeordnet werden können. Allerdings nur insofern perfekt anonymisiert wurde. Bei einer Pseudonymisierung sind die Daten weiterhin personenbezogen.
  \end{solution}
\end{exercise}

\begin{exercise}{Digitalisierung historischer Telefonbücher}
  Ein Berliner Unternehmen möchte 100 Jahre alte, gedruckte Telefonbücher digitalisieren \& im Internet veröffentlichen. Ist dies datenschutzrechtlich zulässig?
  \hint{Lassen Sie bei Ihrer Antwort urheber- \& lizenzrechtliche Aspekte außer Acht.}

  \begin{solution}
    Datenschutzrecht (nach DSGVO bzw. BDSG) ist nur auf lebende, natürliche Personen anwendbar. Es ist davon auszugehen, dass die enthaltenen Daten sind nicht auf solche beziehen. Daher ist die Veröffentlichung der Telefonbücher zulässig.
  \end{solution}
\end{exercise}

\begin{exercise}{Recht auf Auskunft}
  Bei einem Hosting-Provider geht per Post eine Beschwerde über die unberechtigte Speicherung von Daten durch einen Online-Shop ein. Der Beschwerdeführer möchte vom Hosting-Provider des Online-Shops wissen, welche Daten über ihn von dem dort gehosteten Online-Shop zu seiner Person gespeichert werden.\newline
  Formulieren Sie ein Antwortschreiben des Hosting-Providers.

  \begin{solution}
    Sehr geehrter Herr X,\newline
    wir haben Ihre Anfrage erhalten. Wir sind jedoch nicht der Betreiber des Online-Shops, sondern lediglich der Hosting-Provider und somit weisungsgebundener Auftragsverarbeiter (Art. 28 DSGVO). Wir haben daher keine Kenntnis über die Daten, die der Online-Shop über Sie speichert. Bitte wenden Sie sich mit Ihrer Anfrage direkt an den Online-Shop.\newline
    Mit freundlichen Grüßen,\newline
    Hosting-Provider
  \end{solution}
\end{exercise}

\begin{exercise}{Juristische Personen}
  Der Hamburger Verein Tier-Interessen \& ehrenamtliche Rettung von Tieren (kurz TIeR) hat ein Problem mit einem Hundehasser. Dieser hat aus Hass auf den Verein ein Protokoll der letzten Vorstandssitzung auf Facebook veröffentlicht. Kann TIeR von Facebook die Löschung der Daten nach Art. 17 DSGVO verlangen? Begründen Sie Ihre Antwort.

  \begin{solution}
    Nein, DSGVO nur für natürliche Personen (Verein=juristische Person) (Art. 1 I DSGVO).
  \end{solution}
\end{exercise}

\begin{exercise}{Löschung = Datenverarbeitung?}
  Stellt das Löschen von personenbezogenen Daten eine Verarbeitung dar? Begründen Sie anhand der DSGVO.

  \begin{solution}
    Löschen explizit erwähnt als Verarbeitung (Art. 4 2. DSGVO).
  \end{solution}
\end{exercise}

\begin{exercise}{Datenminimierung}
  Bewerten Sie folgende Aussage zur DSGVO: "Der Grundsatz der Datenminimierung verbietet Verarbeitungstätigkeiten, bei denen personenbezogene Daten in großem Umfang erforderlich sind." Ist diese Aussage korrekt? Begründen Sie anhand der DSGVO.

  \begin{solution}
    Nicht korrekt, Datenminimierung erfordert lediglich Zweckbindung, verbietet Verarbeitung großer Datenmengen nicht explizit (Art. 5 I c) DSGVO).
  \end{solution}
\end{exercise}

\begin{eexercises}{Betroffenenrechte}{
    Herrn K wird neuerdings verwehrt, im Onlineshop S auf Rechnung zu bezahlen. Auf Nachfrage nennt S eine erst nach wiederholter Mahnung bezahlte Rechnung aus einer früheren Bestellung als Ursache. Die deshalb schlechte Bonität führe zu einem automatischen Ausschluss von der Zahlung per Rechnung.
  }
  \item Herr K bestreitet die verspätete Zahlung und fordert S auf, die falschen Angaben ab sofort nicht mehr zu nutzen und zu korrigieren. Hat K nach der DSGVO das Recht zu diesen Forderungen? Begründen Sie Ihre Antwort.
  \item Herr K hat nach dem Ärger mit S nun doch genug und verlangt von S die Löschung seiner personenbezogenen Daten. Anstatt zu löschen, anonymisiert S die Daten, um sie noch für Statistiken nutzen zu können. Ist eine Anonymisierung anstelle einer Löschung zulässig? Begründen Sie.
  \item Herr K findet außerdem, dass das Shop-System von S durch den automatischen Ausschluss der Bezahlung auf Rechnung gegen den Art. 22 DSGVO verstoßen hat \& möchte sich bei der Aufsichtsbehörde beschweren. Hat seine Beschwerde Aussicht auf Erfolg? Begründen Sie.
\end{eexercises}

\begin{solutions}
  \item Ja, Richtigkeit \& Aktualität der Daten (Art 5 I d) DSGVO) und Recht auf Berichtigung (Art. 16 DSGVO).
  \item Grundsätzlich Recht auf Löschung (Art. 17 DSGVO), Anonymisierung für statistische Zwecke allerdings ausnahmsweise erlaubt (Art. 17 III d) DSGVO i.V.m. Art. 89 DSGVO).
  \item Grundsätzlich Recht auf nicht ausschließlich automatisierte Einzelfallentscheidung (Art. 22 DSGVO), allerdings Ausnahmen für Vertrag (Art. 22 II a) DSGVO). Daher keine Garantie für, aber Aussicht auf Erfolg.
\end{solutions}

\begin{exercises}{Rechtmäßigkeit der Verarbeitung}
\item In welchen Fällen erlaubt die DSGVO die Verarbeitung von personenbezogenen Daten auch ohne die Einwilligung des Betroffenen? Nennen Sie drei Fälle samt Fundstelle im Gesetz \& beschreiben Sie diese kurz.
\item Um ihrer Schwester eine Freude zu machen, beauftragt Frau K den Blumenhändler B der Schwester am Geburtstag Blumen zu bringen. Dazu teilt K dem Blumenhändler B die Anschrift \& das Geburtsdatum der Schwester mit. Prüfen Sie die Datenverarbeitung durch K \& B hinsichtlich ihrer Rechtmäßigkeit nach der DSGVO. Sechs Punkte genügen.
\end{exercises}

\begin{solutions}
  \item Art. 6 DSGVO
  \begin{itemize}
    \item[b)] Vertrag: Falls notwendig zum Abschluss oder zur Erfüllung eines Vertrags
    \item[c)] Rechtspflicht: Falls notwendig zur Erfüllung einer rechtlichen Verpflichtung
    \item[d)] lebenswichtige Interessen: Falls notwendig zum Schutz lebenswichtiger Interessen des Betroffenen oder einer anderen natürlichen Person
    \item[e)] öffentliches Interesse: Wahrnehmung einer Aufgabe im öffentlichen Interesse (durch EU oder Nationalrecht)
    \item[f)] berechtigtes Interesse: Falls notwendig zur Wahrung berechtigter Interessen des Verantwortlichen oder eines Dritten
  \end{itemize}
  \item Prüfung des Art. 6 DSGVO
  \begin{itemize}
    \item Einwilligung? \xmark
    \item Vertrag? \xmark
    \item Rechtspflicht? \xmark
    \item lebenswichtige Interessen? \xmark
    \item öffentliches Interesse? \xmark
    \item berechtigtes Interesse? K: Freude der Schwester \checkmark
          \begin{itemize}
            \item Grundrechte-/freiheiten eingeschränkt? \xmark
            \item Kind? \xmark
          \end{itemize}
    \item[$\hookrightarrow$] Rechtmäßig, da berechtigtes Interesse von K.
  \end{itemize}
\end{solutions}

\begin{exercise}{Zweckbindung}
  Erklären Sie kurz den Begriff Zweckbindung. Wo wird die Zweckbindung in der DSGVO geregelt und welche Ausnahmen gibt es?

  \begin{solution}
    Zweckbindung: Daten dürfen nur für den Zweck verarbeitet werden, für den sie erhoben wurden.\par
    Art. 5 I b) DSGVO\par
    Ausnahmen (Art. 89 DSGVO): Weitere Verarbeitung für Archivzwecke, wissenschaftliche oder historische Forschungszwecke oder statistische Zwecke.
  \end{solution}
\end{exercise}

\begin{exercises}{Einwilligung}
\item Weshalb kann die Einwilligung als Rechtsgrundlage für die Verarbeitung von Daten eines Angestellten (Betroffener) durch dessen Arbeitgeber (Verantwortlicher) als problematisch angesehen werden?
\item Was wird in Zusammenhang mit einer Einwilligung als Kopplungsverbot bezeichnet?
\item In einem Online-Bestellformular soll ein Kästchen (Checkbox) zur Einwilligung in die Verarbeitung zu Werbezwecken eingebaut werden. Begründen Sie anhand der DSGVO, ob das Kästchen standardmäßig angekreuzt oder leer sein sollte, damit die Einwilligung wirksam ist. Nennen Sie die Fundstelle im Gesetz.
\end{exercises}

\begin{solutions}
  \item Abhängigkeitsverhältnis, Freiwilligkeit fraglich, Widerruf schwer möglich.
  \item Verbot der Verknüpfung der Einwilligung mit anderen Umständen.
  \item Leer, Art. 7 I DSGVO.
\end{solutions}

\begin{eexercises}{Verarbeitung durch Dienstleister}{
    Das Start-up-Unternehmen S geht mit einem neuen, innovativen Telemediendienst auf den Markt. Da es noch über keine eigene Infrastruktur verfügt, beaucragt es Hosting-Dienstleister H mit dem Hosting des Dienstes. Die Administration des Dienstes erfolgt durch Mitarbeiter des Unternehmens S über Fernzugriff. Sowohl S als auch H sitzen in Deutschland.
  }
  \item Liegt hier eine Auftragsdatenverarbeitung oder Funktionsübertragung vor? Begründen Sie. Gehen Sie davon aus, dass personenbezogene Daten verarbeitet werden.
  \item In welchem der beiden Fälle Auftragsdatenverarbeitung oder Funktionsübertragung läge eine Übermittlung im Sinne des BDSG vor? Begründen Sie.
\end{eexercises}

\begin{solutions}
  \item Auftragsdatenverarbeitung, da H weisungsgebunden (Art. 28 DSGVO).
  \item Funktionsübertragung, da S die Administration selbst übernimmt. Relevante Paragraphen des BDSG wurden 2018 in die DSGVO überführt, sind daher nicht mehr anzugeben.
\end{solutions}

\begin{exercise}{Data Protection by Design}
  Art. 25 (1) DSGVO benennt Aspekte, die ein Verantwortlicher bei der Auswahl geeigneter technischer \& organisatorischer Maßnahmen zu berücksichtigen hat. Beschreiben Sie den Einfluss von zwei der genannten Aspekte auf die Eignung einer Maßnahme.

  \begin{solution}
    \begin{itemize}
      \item Stand der Technik: Maßnahmen müssen auf dem neuesten Stand der Technik sein, um Sicherheit zu gewährleisten.
      \item Implementierungskosten: Maßnahmen müssen wirtschaftlich vertretbar sein, um die Umsetzung zu ermöglichen.
      \item Umfang: Maßnahmen müssen auf den Umfang der Verarbeitung abgestimmt sein, um effektiv zu sein.
      \item Umstände: Maßnahmen müssen auf die Umstände der Verarbeitung abgestimmt sein, um effektiv zu sein.
      \item Verarbeitungszweck: Maßnahmen müssen auf den Verarbeitungszweck abgestimmt sein, um effektiv zu sein.
      \item Risiken: Maßnahmen müssen auf die Eintrittswahrscheinlichkeit \& Schwere der Risiken abgestimmt sein, um effektiv zu sein.
    \end{itemize}
  \end{solution}
\end{exercise}

\begin{eexercises}{Safe Harbour}{
    Der Europäische Gerichtshof (EuGH) hat im Oktober 2015 die Entscheidung der EU-Kommission zum "Sicheren Hafen" (Safe Harbour) anulliert.
  }
  \item Was versteht man unter Safe Harbour?
  \item Warum hat der EuGH die Safe Harbour-Entscheidung der EU-Kommission annulliert?
\end{eexercises}

\begin{solutions}
  \item Safe Harbour war ein Abkommen zwischen der EU \& den USA, das den Datentransfer in die USA regelte.
  \item Der EuGH hat die Safe Harbour-Entscheidung der EU-Kommission annulliert, weil die USA keine ausreichenden Datenschutzstandards gewährleisten (insbesondere durch den Patriot Act). Safe Harbour wurde durch das (inzwischen ebenfalls gekippte) Privacy Shield ersetzt.
\end{solutions}

\begin{exercise}{Datenschutzerklärung}
  Sie haben folgende Datenschutzerklärung einer Firma vorliegen. Die Firma bietet eine Reihe von Apps für Handys an. Bei einer Überprüfung der Datenschutzerklärung fallen Ihnen Punkte auf, die unter Berufung auf die DSGVO anfechtbar sind.

  \begin{quote}
    \subsection*{Datenschutzerklärung}
    \subsubsection*{Verantwortlicher}
    Verantwortlicher gemäß Art. 4 Abs. 7 DSGVO ist Datenkrake GmbH \& Co. KG

    \subsubsection*{Allgemeine Informationen zur Datenverarbeitung}
    Einige Daten stellen Sie direkt bereit, andere erhalten wir durch das Sammeln von Informationen über Ihre Aktivitäten, Nutzung und Erfahrungen mit unseren Produkten. Wir erhalten ebenfalls Daten über Sie von Drittanbietern. Wir verarbeiten diese Vertrags- und Zahlungsdaten sowie weitere anfallende Verbindungsdaten und behalten uns eine Weitergabe an Dritte für gewerbliche Zwecke vor.\\
    Viele unserer Produkte nutzen personenbezogene Daten, die Sie über einen Dienst bereitstellen. Wenn Sie sich dazu entscheiden, keine Daten anzubieten, die für eine Bereitstellung eines Produkts oder einer Funktion erforderlich sind, werden Sie möglicherweise nicht in der Lage sein, das Produkt oder die Funktion zu verwenden. Wir müssen ebenfalls personenbezogene Daten gesetzlich sammeln, wenn wir einen Vertrag mit Ihnen unterzeichnen oder eingehen. Wenn Sie keine Daten angeben möchten, können wir mit Ihnen keinen Vertrag eingehen.\\
    Wir verwenden die Daten für unser Unternehmen, inklusive der Analyse und Leistung, der Einhaltung unserer gesetzlichen Verpflichtung, für unsere Weiterentwicklung sowie zur Forschung. Wir kombinieren die erfassten Daten aus verschiedenen Kontexten oder von Drittanbietern, damit wir Ihnen eine vollständige und personalisierte Erfahrung bieten können, um fundierte Entscheidungen zu treffen oder diese zu anderen legitimen Zwecken zu verwenden.

    \subsubsection*{Widerrufs- und Betroffenenrechte}
    Sie können Ihre personenbezogenen Daten kontrollieren, indem Sie sich an den Support wenden, den wir bereitstellen. In einigen Fällen sind Ihre Möglichkeiten zur Verwaltung Ihrer personenbezogenen Daten eingeschränkt. Wir werden innerhalb von 30 Tagen auf Anfragen reagieren, um Ihre personenbezogenen Daten zu verwalten.\\
    Sie können der Nutzung Ihrer personenbezogenen Daten widersprechen, allerdings können Sie unsere Dienste dann nicht mehr oder nur noch mit Einschränkungen nutzen.\\
    Bei einer Widerrufung der Einwilligung der von uns erhobenen Daten löschen wir Ihre Daten nach einer Bearbeitungsfrist.
    \vspace{1em}\\
    Mit der Einwilligung stimmen Sie dem Vertrag und der Nutzung zu sowie der Weitergabe und Verarbeitung Ihrer Daten von uns und an Dritte.
  \end{quote}

  Bitte nennen Sie fünf Punkte, die Ihrer Meinung nach nicht erfüllt sind oder gegen die verstoßen wird, jeweils mit Benennung des Artikels. Erläutern Sie zudem warum und gegen was verstoßen wird oder was nicht erfüllt wird.

  \begin{solution}
    % Alle Punkte aufzählen
    \begin{itemize}
      \item (Verantwortlicher) Art. 13 I a: Anschrift des Verantwortlichen fehlt.
      \item (Allgemeine Informationen) Art. 5 I b: Zwecke sind nicht eindeutig bestimmt (zB. "unsere Weiterentwicklung" ist zu unbestimmt).
      \item (Allgemeine Informationen) Art. 5 I c: Datenminimierung nicht gewährleistet, da unverhältnismäßig viele Daten verarbeitet werden.
      \item (Allgemeine Informationen) Art. 7 IV: Einwilligung ist an Vertrag gekoppelt.
      \item (Allgemeine Informationen) Art. 13 I e: Empfänger der Daten werden nicht konkretisiert.
      \item (Widerrufs- \& Betroffenenrechte) Art. 7 III: Widerruf der Einwilligung ist zu restriktiv.
      \item (Widerrufs- \& Betroffenenrechte) Art. 15 I: Recht auf Auskunft nicht gewährleistet.
    \end{itemize}
  \end{solution}
\end{exercise}

\begin{exercise}{Privacy Design Strategies}
  Bitte nennen Sie je ein Beispiel für die folgenden Privacy Design Strategies im Kontext von Big Data: Demonstrate, Separate, Enforce, Inform, Hide, Minimise, Perturbate, Aggregate, Control.
  \hint{Bitte geben Sie je eine Antwort für diese neun Privacy Design Strategies.}

  \begin{solution}
    \begin{itemize}
      \item Demonstrate: Bericht über Datenschutzmaßnahmen veröffentlichen
      \item Separate: Trennung von Posts \& Profilinformationen
      \item Enforce: Passwortrichtlinien
      \item Inform: Datenschutzerklärung bereitstellen
      \item Hide: Profildaten anonymisieren
      \item Minimise: Ungenutzte Nutzungsdaten nach 30 Tagen löschen
      \item Perturbate: Zufällige "tote Profile" in Datenbank
      \item Aggregate: Zugriffsdaten nach Ländern zusammenfassen
      \item Control: Opt-out für personalisierte Werbung
    \end{itemize}
  \end{solution}
\end{exercise}

\exercisesheet[2020]{Erstttermin}
\begin{exercises}{Löschung}
\item Herr D schickt zum Valentinstag allen Ex-Partnern, darunter auch Frau E, eine Postkarte. Da Frau E nichts mehr mit D zu tun haben will, schreibt sie eine Postkarte an D \& verlangt nach Art. 17 DSGVO die Löschung ihrer Adresse. Muss D dem Folge leisten?
\item Stellt das Löschen von personenbezogenen Daten ein Verarbeitung dar? Begründen Sie anhand DSGVO.
\end{exercises}

\begin{eexercises}{Anonymität}{
    Krankenhaus A gibt angeblich anonymisierte Daten seiner Patienten zu Forschungszwecken an andere medizinische Institute. Die Daten umfassen folgende Angaben über den Patienten: Krankheitsbild, Zeitraum der Krankheit, Geburtsjahr, Postleitzahl.
  }
  \item Wo \& wie sind in der DSGVO die Kriterien für Anonymität geregelt?
  \item Ist der Datensatz hinreichend anonymisiert? Begründen Sie.
\end{eexercises}

\begin{eexercises}{Rechtmäßigkeit der Verarbeitung}{
    Das Unternehmen U organisiert zum Valentinstag ein Dating-Event. Bei der Anmeldung geben die Teilnehmer U Zugriff auf ihr privates Social Media Profil, um gemeinsame Interessen zu finden. Nachdem U anhand der Daten die sexuelle Orientierung der Teilnehmer berechnet, sind einige Teilnehmer empört.
  }
  \item War die Ermittlung der sexuellen Orientierung zulässig? Begründen Sie.
  \item Wie ändert sich die Zulässigkeit, wenn öffentliche Profile ohne Einwilligung genutzt werden?
  \item Spielt die Richtigkeit der berechneten sexuellen Orientierung eine Rolle für die Zulässigkeit?
\end{eexercises}

\exercisesheet[20211]{Ersttermin}
\begin{eexercises}{Systematik}{
    Sie haben die Aufgabe, zu prüfen, ob die Datenverarbeitung einer Webseite der DSGVO zuwiderläuft.
  }
  \item Ordnen Sie die folgenden 3 Begriffe nach der Reihenfolge Ihres Vorgehens bei der Prüfung: Anwendungsbereich, Informationspflichten, Rechtmäßigkeit der Verarbeitung. Nennen Sie die dazugehörigen Artikel \& begründen Sie die Reihenfolge.
  \item Ordnen Sie die folgenden 6 Phrasen den obigen Schritten zu: Ausübung ausschließlich persönlicher Tätigkeiten, Direkterhebung, Einwilligung, klare \& einfache Sprache, Marktortprinzip, Vertragserfüllung. Nennen Sie jeweils eine möglichst präzise dazugehörige Gesetzesstelle.
\end{eexercises}

\begin{eexercises}{Räumlicher Anwendungsbereich}{
    Das Unternehmen EarView mit alleinigem Sitz auf den Osterinseln (Chile) bietet einen Service, um Menschen anhand von Ohrabdrücken biometrisch in Fotos im Internet wieder zu erkennen. Dazu sucht EarView nach öffentlich zugänglichen Fotos mit sichtbaren Ohren \& speichert die URL jedes Fotos zusammen mit einem biometrischen Ohrenabdruck. Alle Bilder mit hinreichend ähnlichem Ohrenabdruck bilden ein Ohrenprofil. Dabei sammelt EarView auch Ohrenprofile von Betroffenen in der EU. Der Service wird nur staatlichen Polizeibehörden \& Geheimdiensten angeboten.
  }
  \item Ist die Tätigkeit EarViews Verhaltensbeobachtung im Sinne der DSGVO? Begründen Sie.
  \item Prüfen Sie die räumliche Anwendbarkeit der DSGVO \& wie sich Verhaltensbeobachtung darauf auswirkt.
  \item Muss EarView die Betroffenen über die Datenverarbeitung informieren? (die DSGVO findet Anwendung)
\end{eexercises}

\begin{eexercises}{Faktische Anonymität}{
    In der Rechtsdogmatik wird zwischen faktischer \& absoluter Anonymität unterschieden.
  }
  \item Erläutern Sie die Unterscheidung.
  \item Nennen Sie ein Beispiel, wie durch technischen Fortschritt ein faktisch anonymisiertes Datum wieder personenbezogen werden kann.
\end{eexercises}

\exercisesheet[20212]{Zweittermin}
\begin{eexercises}[2]{Datenschutzerklärung}{
    Bewerten Sie die folgenden Aussagen zur DSGVO. Sind diese korrekt? Begründen Sie jeweils anhand der DSGVO.
  }
  \item "Der Grundsatz der Datenminimierung verbietet Verarbeitungstätigkeiten, bei denen personenbezogene Daten in großem Umfang erforderlich sind."
  \item "Das Anonymisieren von personenbezogenen Daten bedarf einer Rechtsgrundlage nach Art. 6."
\end{eexercises}

\begin{eexercises}{Betroffenenrechte}{
    Patient P braucht eine lebensnotwendige Blutspende, hat aber eine äußerst seltene Blutgruppe. Ärzen A erinnert sich, dass sie auch Ps 14-jährigen Cousin C behandelt hat \& testet dessen Blutprobe auf Übereinsemmung mit P.
  }
  \item Erläutern Sie, ob die Ärzten nach der DSGVO dazu eine Einwilligung von C häYe einholen müssen.
  \item Cousin C möchte seine Patientendaten nun von As Praxis an ein Ahnenforschungsinstitut übertragen lassen. Nennen \& erläutern Sie zwei Gründe, warum C eine Datenweitergabe nicht nach Art. 20 einfordern kann.
  \item Gibt es (auch) andere Betroffenenrechte, mit denen C seine Daten erhalten kann?
  \item Das Ahnenforschungsinstitut lässt sich von C eine Einwilligung für die Datenverarbeitung geben. Was muss das Institut – außer Art. 7 – beachten, damit in Cs Fall eine Einwilligung wirksam ist?
\end{eexercises}

\exercisesheet[2022]{Ersttermin}
\begin{eexercises}{DSGVO}{
    Bewerten Sie die folgenden Aussagen zur DSGVO. Sind diese korrekt? Begründen Sie jeweils anhand der DSGVO.
  }
  \item "Personenbezogene Daten dürfen nicht ohne Einwilligung verarbeitet werden."
  \item "Der Grundsatz der Datenminimierung verbietet Verarbeitungstätigkeiten, für die personenbezogene Daten in großem Umfang erforderlich sind."
  \item "Verantwortliche müssen keine Daten zur Identifizierung einer Person (beispielsweise bei Anfragen nach Art. 15) erheben, wenn diese ansonsten nicht erforderlich wären."
\end{eexercises}

\begin{exercise}{DSGVO Konformität}
  Sie beraten einen Hersteller von Software für virtuelle Klassenzimmer im Hinblick auf das Datenschutzrecht. H betreibt die Software nicht selbst, sondern verkauft sie an Schulen \& Bildungseinrichtungen zum Betrieb vor Ort. Bezüglich einer neuen Funktionalität in der kürzlich veröffentlichten Version möchte H Ihre Einschätzung, ob diese aus Sicht der DSGVO problematisch sein könnte:
  \begin{quote}
    Das virtuelle Klassenzimmer protokolliert automatisch die aktive Teilnahme der Schüler am Unterricht (zB. Anzahl der gestellten Fragen pro Schüler) \& gibt den Lehrenden darüber Bericht.
  \end{quote}

  \begin{enumerate}
    \item H meint, die Lehrenden könnten ja auch vorher schon alle Aktivitäten manuell erfassen \& Notizen über Schüler machen. Erläutern Sie, worin für die Anwendbarkeit der DSGVO rechtlich der Unterschied zwischen der manuellen Notiz \& Protokollierung durch die Software besteht.
  \end{enumerate}

  Die neue Funktionalität berechnet zudem pro Schüler einen Score-Wert für die Mitarbeit. Darin ließen Faktoren wie die individuelle Sprechzeit \& die Häufigkeit verschiedener Aktionen mit ein.

  \begin{enumerate}
    \setcounter{enumi}{1}
    \item Nehmen Sie an, Lehrende vergeben anhand dieses Score-Wertes Mitarbeitsnoten. Unter welchen Bedingungen wäre diese Notenvergabe aus Sicht der DSGVO unzulässig? Begründen Sie.
    \item Der Hersteller hat nun Zweifel, ob die Software so DSGVO-konform ist. Muss er deshalb Sorge haben, dass ihm Geldbußen nach Art. 83 Abs. 5 auferlegt werden? Erläutern Sie dazu anhand der DSGVO, ob H den Pflichten des Gesetzes unterliegt.
  \end{enumerate}
\end{exercise}

\begin{eexercises}{Datenschutzbeauftragter}{
    Sie sind Datenschutzbeauftragter bei einem Online-Versandhändler. Per E-Mail erreicht Sie eine Auskunftsanfrage nach Art. 15 DSGVO. Der Absender ist namentlich als Kunde bekannt, jedoch ist im Kundenkonto eine andere E-Mail-Adresse hinterlegt. Zusätzlich hat das Kundenkonto die Wohnungsanschrift des Kunden als Rechnungsadresse gespeichert.
    \hint{Es handelt sich um personenbezogene Daten}
  }
  \item Um die Anfrage schnell vom Tisch zu haben, überlegen Sie, die Daten, die Ihr System zu diesem Kundennamen führt, an die E-Mail-Adresse des Anfragenden zu schicken. Wäre dies ein Verstoß gegen Pflichten der DSGVO? Begründen Sie.
  \item Weil Sie an der Identität des Anfragenden Zweifel haben, erwägen Sie, die Anfrage einfach zu ignorieren. Wäre das zulässig? Begründen Sie.
  \item Sie fordern zur Identitätsbestätigung vom Anfragenden eine Kopie eines amtlichen Ausweisdokumentes an, das dessen Wohnanschrift ausweist. Dürfen Sie das? Gehen Sie auf die Erforderlichkeit eines amtlichen Dokumentes ein.
\end{eexercises}

\begin{exercise}{Privacy by Design: Strategien}
  Aus der Vorlesung kennen Sie fünf technische Strategien zum Schutz personenbezogener Daten: Aggregate, Hide, Minimise, Perturbate \& Separate.
  Erläutern Sie jede dieser Strategien knapp \& geben Sie jeweils zusätzlich ein konkretes Beispiel.
\end{exercise}

\exercisesheet[2023]{Ersttermin}
\begin{eexercises}{Räumlicher Zuständigkeitsbereich}{
    Kunde K kauft bei einem Online-Shop ein, dieser nutzt einen Hosting-Anbieter (z.B. Microsoft Azure) X, welcher sich in den USA befindet. Außerdem wird ein Logistikdienstleiter L genutzt um Pakete nach dem Erwerb zu versenden.
  }
  \item K will eine Auskunft vom Online-Shop über die genutzten Daten, was muss die Antwort auf diese Auskunftsanfrage enthalten \& in welchem Artikel ist sie definiert?
  \item K will außerdem eine Anfrage an L senden, muss L auf diese Anfrage antworten? Gehen Sie hierauf \& auf die verschiedenen möglichen Verhältnisformen zwischen L \& dem Online-Shop ein \& nennen die entsprechenden Artikel.
\end{eexercises}

\begin{exercises}{Datenschutzverletzungen}
\item In welchen Fällen muss ein Verantwortlicher im Falle einer Datenschutzverletzung diese melden?
\item Drei Beispiele einer Datenschutzverletzung gegeben. Begründen Sie nach der Definition in a), ob diese meldungspflichtig sind.
\item Was für Präventivmaßnahmen können getroffen werden um Datenschutzverletzungen vorzubeugen? Nennen Sie zwei.
\item Was für Maßnahmen müssen nach einer Datenschutzverletzung durchgeführt werden? Nennen Sie zwei.
\end{exercises}

\begin{eexercises}{DSGVO-Konformität}{
    Eine Website soll auf Konformität zur DSGVO überprüft werden.
  }
  \item Sortieren Sie die folgenden Punkte nach einer sinnvollen Reihenfolge der Überprüfung \& begründen Sie diese: Informationspflichten, Anwendungsbereich, Rechtmäßigkeit der Datenverarbeitung
  \item Ordnen Sie die folgenden fünf Begriffe den unterschiedlichen Aspekten aus 1. zu und nennen möglichst genau die dazugehörige Stelle der DSGVO: berechtigtes Interesse, Kopplungsverbot, persönliche Tätigkeiten, einfache \& verständliche Sprache, Erhebung aus dritter Quelle
\end{eexercises}

\end{document}