\documentclass{article}

\usepackage{summary}

\subject{Modellierung \& Analyse komplexer Systeme}
\semester{Winter 2023/24}
\author{Leopold Lemmermann}

\begin{document}\createtitle

\section{Automaten}
\subsection{Potenz}
$A^2$: Umwandlung von NFAs zu DFAs

\begin{enumerate}
  \item Powerset über States $\mathcal{P}(S)$ bilden (alle Kombinationen inkl. $\epsilon$)
  \item State $\times$ Transition Tabelle für neue Übergangsrelation $\delta$
\end{enumerate}

\subsection{Komplement}
$\bar{A}$: für $L_{\bar{A}} =\Sigma^* \setminus L_A$

\begin{enumerate}
  \item $A$ zu $DFA$ transformieren
  \item alle Endstates invertieren: $F_{\bar{A}}=Q_A\setminus F_A$
\end{enumerate}

\subsection{Konkatenation}
$A\cdot B$: für $L_{A\cdot B}=L_A\cdot L_B$

\begin{enumerate}
  \item Wenn $\epsilon\in L_A$, dann $Q_{A\cdot B}^0=Q_A^0+Q_B^0$, sonst $Q_{A\cdot B}=Q_A^0$
  \item Wenn Kante von $q\in Q_A$ zu $q_f\in F_A$, dann $\forall q_0\in Q_B^0$: Kante von $q$ nach $q_0$ ziehen.
  \item $F_{a\cdot B}=F_B$
\end{enumerate}

\subsection{Produkt}
$A\cap B$: Durchschnitt bilden für $L_{A\cap B}=L_A\cap L_B$

\begin{enumerate}
  \item kartesisches Produkt der Zustände bilden $Q=Q_A\times Q_B$
  \item Zustände synchronisieren
  \item[Büchi:] zusätzliche Komponente um mehrere Endzustände zu garantieren.
\end{enumerate}



\section{Prozesse}

\begin{enumerate}
  \item[Symbole]
        \begin{itemize}
          \item P/T-Netz $N = (P,T,F,W,m_0)$
          \item (Verzweigungs-)prozess $R=(B,E,F_R)$: (Ereignis-/)Kausalnetz
          \item Abbildung $\Phi =(\Phi_P,\Phi_T)$
        \end{itemize}
  \item Passende Struktur: $(x,y)\in F_R \to (\Phi(x), \Phi(y))\in f$
        \begin{quote}
          Alle Elemente eines Prozesses bilden Elemente des Netzes ab.
        \end{quote}
  \item Stellen beschreiben Initialmarkierung: $\forall p\in P: m_0(p)=|\Phi^{-1}(p)\cap \circ R|$
        \begin{quote}
          Die Initialmarkierung eines Netzplatzes entspricht der Anzahl der Abbildungen dieses Platzes im minimalen Bereich (vor Schaltungen bzw. Transitionen) des Prozesses.
        \end{quote}
  \item Verträgliche Kantengewichtungen: …
        \begin{quote}
          Sowohl Vor-, als auch Rückkanten von Transitionen des Prozesses, entsprechen anzahlsmäßig den Kantengewichtungen zu den Netzplätze.
        \end{quote}
\end{enumerate}

\section{Überdeckungsgraphen}

\begin{algorithm}[H]
  \caption{\texttt{drawCoverage($N$)}}

  \KwData{A P/T net $N$}
  \KwResult{A coverage graph}
  \BlankLine

  \While{uncolored nodes in $V$}{
    choose next node and color\;
    \ForEach{active transition}{
      replace larger fields in strictly larger successive marking with $\omega$\;
      \lIf{successive marking not in coverage graph}{add successive marking to coverage graph}
      add transition edge from marking to successive marking in coverage graph
    }
  }
\end{algorithm}

\section{CTL-Algorithmus}

Der CTL-Algorithmus formalisiert das Model-Checking über CTL statements.

\begin{algorithm}[H]
  \caption{\texttt{check($f$)}}

  \KwData{a CTL statement $f$}
  \KwResult{States are labeled with whether they satisfy $f$}
  \BlankLine

  \Switch{$f$}{
    \Case{$\in AP$}{
      \ForEach{$s\in S$}{
        \lIf{$f\in E_S(s)$}{$s.labels\gets s.labels\cup f$}
      }
    }
    \Case{$\lnot f_1$}{
      \texttt{check($f_1$)}\;
      \ForEach{$s\in S$}{\lIf{$f_1\not\in s.labels$}{$s.labels\gets s.labels\cup \lnot f_1$}}
    }
    \Case{$f_1\lor f_2$}{
      \texttt{check($f_1$)}; \texttt{check($f_2$)}\;
      \ForEach{$s\in S$}{\lIf{$f_1\in s.labels\lor f_2\in s.labels$}{$s.labels\gets s.labels\cup f_1\lor f_2$}}
    }
    \Case{$EXf_1$}{
      \texttt{check($f_1$)}\;
      \ForEach{$s\in S$}{\lIf{$\exists t\in R(s): f_1\in s.labels$}{$s.labels\gets s.labels\cup EXf_1$}}
    }
    \Case{$E[f_1\textbf{U}f_2]$}{
      \texttt{check($f_1$)}; \texttt{check($f_2$)}\;
      \texttt{checkEU($f_1,f_2$)}\explain*{mark all $f_2$ states, traverse backwards and mark $f_1$ states with $E[f_1\textbf{U}f_2]$}
    }
    \Case{$EGf_1$}{
      \texttt{check($f_1$)}\;
      \texttt{checkEG($f_1$)}\explain*{find SCCs (cycles) in $f_1$ states, label states in SCCs and leading to SCCs with $EGf_1$}
    }
  }

\end{algorithm}


\end{document}