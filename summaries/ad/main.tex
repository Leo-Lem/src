\documentclass{article}

\usepackage{summary}

\subject{Algorithms and Data Structures}
\semester{Winter 2023/24}
\author{Leopold Lemmermann}

\begin{document}\createtitle

\section{The Asymptotic View}
\subsection{Bachmann-Landau Notation}
\begin{itemize}
  \item[$\bigO$] (big O: upper bound) $f\in \bigO(g) \Leftrightarrow ^{\exists c > 0}_{\exists n_0 > 0}: f \leq c \cdot g \forall n > n_0$ \textbf{or} $\limsup_{n\to\infty} \frac{f}{g} < \infty$
  \item[$\Omega$] (big Omega: lower bound) $f\in \Omega(g) \Leftrightarrow ^{\exists c > 0}_{\exists n_0 > 0}: f \geq c \cdot g \forall n > n_0$ \textbf{or} $\liminf_{n\to\infty} \frac{f}{g} > 0$
  \item[$\littleO$] (little o: strict upper bound) $f\in \littleO(g) \Leftrightarrow ^{\forall c > 0}_{\exists n_0 > 0}: f < c \cdot g \forall n > n_0$ \textbf{or} $\lim_{n\to\infty} \frac{f}{g} = 0$
  \item[$\omega$] (little omega: strict lower bound) $f\in \omega(g) \Leftrightarrow ^{\forall c > 0}_{\exists n_0 > 0}: f > c \cdot g \forall n > n_0$ \textbf{or} $\lim_{n\to\infty} \frac{f}{g} = \infty$
  \item[$\Theta$] (big Theta: tight bound) $f\in \Theta(g) \Leftrightarrow f\in \bigO(g) \land f\in \Omega(g)$
\end{itemize}

\subsection{Master Theorem}
The master theorem is a method for solving recurrences of the form
\begin{equation}
  T(n) = \begin{cases}
    \bigO(1)                                                                                   & \forall n \leq n_0 \\
    \underbrace{aT(\frac{n}{b})}_{\text{recursion}} + \underbrace{f(n)}_{\text{non-recursion}} & \text{else}
  \end{cases}
\end{equation}
where $a \geq 1$, $b > 1$, $f(n)$ is a given function, and $n_0$ is a given constant.

\begin{itemize}
  \item[I.] (recursion dominates) $\exists \epsilon > 0: f(n) \in \bigO(n^{\log_b(a)-\epsilon}) \Rightarrow T(n) \in \Theta(n^{\log_b a})$
  \item[II.] (recursion and non-recursion are balanced) $\exists k \ge 0: f(n) \in \Theta(n^{\log_b a}\log^k n) \Rightarrow T(n) \in \Theta(n^{\log_b a} \log^{k+1} n)$
  \item[III.] (non-recursion dominates) $\underbrace{^{\exists \epsilon > 0: f(n) \in \Omega(n^{\log_b(a)+\epsilon})}_{\exists c \in [0\dots1]: a f(\frac{n}{b}) \leq c f(n)}}_{\text{\tiny regularity condition}} \Rightarrow T(n) \in \Theta(f(n))$
\end{itemize}



\section{Sorting}
\subsection{Elementary sorting algorithms}
The elementary sorting algorithms \algref{selectionSort} \algref{insertionSort} and \algref{bubbleSort} have a time complexity of $\bigO(n^2)$. They sort stable and in-place.
\subsubsection{Selection Sort}
Continue selecting the smallest element and swapping it with the first unsorted element.\par
\begin{alg}
  \signed{selectionSort}{A}{A_\leq}
  \params{An array $A$ of $n$ elements}{The array $A$ sorted in non-decreasing order}

  \For{$i \gets 1$ \KwTo $n-1$}{
    $min \gets i$\;
    \For{$j \gets i+1$ \KwTo $n$}{
      \If{$A[j] < A[min]$}{
        $min \gets j$\;
      }
    }
    swap $A[i]$ and $A[min]$\;
  }
\end{alg}


\subsubsection{Insertion Sort}
Continue inserting the first unsorted element into the sorted part of the array.\par
\begin{alg}
  \signed{insertionSort}{A}{A_{\leq}}
  \params{An array $A$ of $n$ elements}{The array $A$ sorted in non-decreasing order}

  \For{$i \gets 2$ \KwTo $n$}{
    $key \gets A[i]$\;
    $j \gets i-1$\;
    \While{$j > 0$ \textbf{and} $A[j] > key$}{
      $A[j+1] \gets A[j]$\;
      $j \gets j-1$\;
    }
    $A[j+1] \gets key$\;
  }
\end{alg}

\subsubsection{Bubble Sort}
Continue swapping adjacent elements if they are in the wrong order.\par
\begin{algorithm}[H]
  \caption{\textalgo{bubbleSort($A$)}}\label{alg:bubblesort}

  \KwData{An array $A$ of $n$ elements}
  \KwResult{The array $A$ sorted in non-decreasing order}
  \BlankLine

  \For{$i \gets 1$ \KwTo $n-1$}{
    \For{$j \gets 1$ \KwTo $n-i$}{
      \If{$A[j] > A[j+1]$}{
        swap $A[j]$ and $A[j+1]$
      }
    }
  }
\end{algorithm}

\subsection{Divide \& Conquer}
The divide and conquer sorting algorithms \algref{mergeSort} and \algref{quickSort} have a time complexity of $\bigO(n\log n)$.
\subsubsection{Merge Sort}
\textbf{Idea}: Divide the array into two halves, sort the halves, and then merge them.\par
\textbf{Properties}: Sorts stable and not in-place.\par
\begin{alg}
  \signed{mergeSort}{A, p, r}{A_\leq}
  \params{An array $A$ of $n$ elements, and indices $p$ and $r$}{The subarray $A[p \dots r]$ sorted in non-decreasing order}

  \If{$p < r$}{
    $q \gets \lfloor\frac{p+r}{2}\rfloor$\;
    \callalg{mergeSort}{A,p,q}
    \callalg{mergeSort}{A,q+1,r}
    \textalgo{merge($A$, $p$, $q$, $r$)}\;
  }
\end{alg}

\subsubsection{Quick Sort}
\textbf{Idea}: Choose a pivot element and partition the array into elements smaller and larger than the pivot. Then sort the partitions.\par
\textbf{Properties}: Sorts in-place, but not stable.\par
\begin{alg}
  \signed{Partition}{A, l, r}{\empty}
  \params{$l$ - linkes Blockende, $r$ - rechtes Blockende, $A$ - Array}{\empty}
  $i \gets l - 1$\;
  \For{$j \gets l$ \KwTo $r - 1$}{
    \If{$A[j] \leq A[r]$}{
      $i \gets i + 1$\;
      swap $A[i], A[j]$\;
    }
  }
  swap $A[i + 1], A[r]$\;
  \KwRet{$(i + 1)$}\;
\end{alg}

\subsection{Heap Sort}
\textbf{Idea}: Use heap structure to sort the array.\par
\textbf{Properties}: Sorts in-place, but not stable.\par
\textbf{Complexity}: The time complexity is $\bigO(n\log n)$.\par
\begin{alg}
  \signed{heapsort}{A}{\empty}

  \callref{buildheap}{A}
  \For{$i \gets length(A)$ \textbf{downto} 2}{
    \call{deletemax}{A}
  }
\end{alg}

\begin{alg}
  \signed{buildheap}{A}{\empty}

  $n \gets length(A)$\;
  \For{$i \gets \lfloor n/2\rfloor$ \textbf{downto} 1}{
    \call{heapifyDown}{A, i}
  }
\end{alg}

\subsection{Non-comparing Sorting}
\subsubsection{Trivial Sort}
\textbf{Assumption}: Small range of input, pairwise different, integers.\par
\textbf{Idea}: Use a lookup table to sort the input.\par
\textbf{Complexity}: Timewise $\bigO(n)$, but requires $\bigO(k)$ space, where $k$ is the range of the input.\par
\begin{alg}
  \signed{trivialSort}{A}{A_\leq}
  \params{An array $A$ of $n$ elements}{The array $A$ sorted in non-decreasing order}

  \lFor{$i \gets 1$ \KwTo $n$}{
    $B[A_i] \gets A_i$
  }
\end{alg}

\subsubsection{Counting Sort}
\textbf{Assumption}: Integers up to $k\in\bigO(n)$.\par
\textbf{Idea}: Count the number of occurrences of each element and then reconstruct the sorted array.\par
\textbf{Complexity}: Timewise $\bigO(n+k)$, where $k$ is the range of the input.\par
\begin{alg}
  \signed{countingSort}{A, B, k}{A_{\leq}}
  \params{An array $A$ of $n$ elements, with indices $1$ to $n$, and an array $B$ to hold the sorted output, with indices $1$ to $n$, and an integer $k$ such that $0 \leq A[i] \leq k$ for all $i$}{The array $A$ sorted in non-decreasing order}

  \For{$i \gets 0$ \KwTo $k$}{
    $C[i] \gets 0$\;
  }
  \For{$j \gets 1$ \KwTo $n$}{
    $C[A[j]] \gets C[A[j]] + 1$\;
  }
  \For{$i \gets 1$ \KwTo $k$}{
    $C[i] \gets C[i] + C[i-1]$\;
  }
  \For{$j \gets n$ \textbf{down to} $1$}{
    $B[C[A[j]]] \gets A[j]$\;
    $C[A[j]] \gets C[A[j]] - 1$\;
  }
\end{alg}

\subsubsection{Radix Sort}
\textbf{Assumption}: Integers of base $m$ with $d$ digits lesser than $m-1$.\par
\textbf{Idea}: Sort the elements by their digits from the least significant to the most significant.\par
\textbf{Complexity}: Timewise $\bigO(d(n+m))$, where $d$ is the number of digits and $m$ is the base of the integers.\par
\begin{algorithm}[H]
  \caption{\textalgo{radixSort($A$, $d$)}}\label{alg:radixsort}

  \KwData{An array $A$ of $n$ elements, each with $d$ digits}
  \KwResult{The array $A$ sorted in non-decreasing order}
  \BlankLine

  \For{$i \gets 1$ \KwTo $d$}{
    use a stable sort to sort array $A$ on digit $i$\;
  }
\end{algorithm}

\subsubsection{Bucket Sort}
\textbf{Assumption}: Uniformly distributed numbers in $[0,1)$.\par
\textbf{Idea}: Distribute the elements into $n$ buckets and then sort each bucket.\par
\textbf{Complexity}: Timewise $\bigO(n^2)$, but $\bigO(n)$ with a good choice of $n$.\par
\begin{algorithm}[H]
  \caption{\textalgo{bucketSort($A$)}}\label{alg:bucketsort}

  \KwData{An array $A$ of $n$ elements}
  \KwResult{The array $A$ sorted in non-decreasing order}
  \BlankLine

  $n \gets \text{length}(A)$\;
  \For{$i \gets 1$ \KwTo $n$}{
    create a new list $B[i]$\;
  }
  \For{$i \gets 1$ \KwTo $n$}{
    insert $A[i]$ into list $B[\lfloor n \cdot A[i] \rfloor]$\;
  }
  \For{$i \gets 1$ \KwTo $n$}{
    sort list $B[i]$\;
  }
  concatenate the lists $B[1], B[2], \ldots, B[n]$ together in order\;
\end{algorithm}



\section{Searching}
\subsection{Red-Black Trees}
\subsubsection{Properties}
\begin{itemize}
  \item[I.]\label{properties:1} [Colored] Every node is either red or black.
  \item[II.]\label{properties:2} [Black Root] The root is black.
  \item[III.]\label{properties:3} [Black Leaves] Every leaf (NIL) is black.
  \item[IV.]\label{properties:4} [Red $\to$ Black Children] If a red node has children, then the children are black.
  \item[V.]\label{properties:5} [Black Height] For each node, all simple paths from the node to descendant leaves contain the same number of black nodes.
\end{itemize}

\subsubsection{Insertion}
\begin{alg}
  \signed{insert}{T, x}{T}
  \params{A red-black tree $T$ and a node $x$}{A red-black tree $T$ with node $x$}

  \comment{find insertion point (and stop at guard/after leaf)}
  $next = T.root$\;
  \Repeat{$next$ is  a leaf}{
    \lIf{$x < next$}{$next \gets next.left$}
    \lElse{$next \gets next.right$}
  }

  \comment{connect $x$ to parent}
  $x.parent \gets next.parent$\;
  \lIf{$x.parent = nil$}{$T.root \gets x$}
  \lElseIf{$x < x.parent$}{$x.parent.left \gets x$}
  \lElse{$x.parent.right \gets x$}

  \comment{fix the properties}
  color $x$ red\;
  \textalgo{fixupAfterInsert($T, x$)}\;
\end{alg}

\begin{alg}
  \signed{fixupAfterInsert}{T, x}{T}
  \params{A red-black tree $T$ and a newly inserted node $x$}{A fixed red-black tree $T$}
  \BlankLine

  \comment*[h]{fix I}\explain*{$x.parent$ is left child in cases 1-3, right child in cases 4-6}
  \While{$x.parent$ is red}{
    \If(\explain*[f]{cases 1 and 4 (only differ in uncle)}){$x.uncle$ is red}{
      color $x.parent$ and $x.uncle$ black\\
      color $x.grandparent$ red\\
      \textbf{continue} with $x \gets x.grandparent$
    }
    \Else(\comment*[h]{if $x.uncle$ is black}){
      \If(\explain*[f]{case 2 $|$ $x$ is left child for case 5}){$x$ is a right child}{
        rotate $x.parent$ left\explain*{rotate right for case 5}
        \textbf{continue} with $x \gets x.parent$
      }
      \Else(\tcp*[h]{if $x$ is left child}\explain*[f]{case 3 $|$ $x$ is right child for case 6 }){
        color $x.parent$ black\\
        color $x.grandparent$ red\\
        rotate $x.grandparent$ right\explain*{for case 6 rotate left}
        \textbf{continue} with $x$
      }
    }
  }

  \comment{fix II}
  color $T.root$ black
\end{alg}

\subsubsection{Deletion}
\begin{algorithm}[H]
  \caption{\texttt{delete(T, x)}}

  \KwData{A red-black tree $T$ and a node $x$}
  \KwResult{A red-black tree $T$ without node $x$}
  \BlankLine

  \comment{swap x with minimum}
  \If{$x$ has two children}{
    $swappee \gets $ minimum$(x.right)$\explain*{alternative: successor(x)}
    $x.key \gets swappee.key$\;
    $x \gets swappee$
  }
  \comment{remove x (or swappee)}
  connect $x.child$ with $x.parent$ ?? $T.root$\explain*{use $T.root$ if $x.parent$ is $nil$}
  \lIf(\explain*[f]{$x$ is red $\to$ no violations}){$x$ is black}{\texttt{fixupAfterDelete($T, x.child$)}}
  \textbf{delete} $x$\;
\end{algorithm}

\begin{algorithm}[H]
  \caption{\texttt{fixupAfterDelete(T, x)}}

  \KwData{A red-black tree $T$ and a node $x$}
  \KwResult{A fixed red-black tree $T$}
  \BlankLine

  \comment*[h]{fix V}\explain*{$x.parent$ is left child in cases 1-4, right child in cases 5-8}
  \While{$x \neq T.root$ \textbf{and} $x$ is black}{
    \If(\explain*[f]{cases 1 and 5 (only differ in sibling)}){$x.sibling$ is red}{
      color $x.sibling$ black\\
      color $x.parent$ red\\
      rotate $x.parent$ left\\
      \textbf{continue} with $x \gets x.sibling$
    }
    \Else(\comment*[h]{if $x.sibling$ is black}){
      \If(\explain*[f]{cases 2 and 6 (only differ in sibling)}){$x.sibling.children$ are black}{
        color $x.sibling$ red\\
        \textbf{continue} with $x \gets x.parent$
      }
      \Else(\comment*[h]{if $x.sibling$ has red child}){
        \If(\explain*[f]{case 3 $|$ $x.sibling.left$ for case 7}){$x.sibling.right$ is black}{
          color $x.sibling.left$ black\explain*{$x.sibling.right$ for case 7}
          color $x.sibling$ red\\
          rotate $x.sibling$ right\explain*{rotate left for case 7}
          \textbf{continue} with $x \gets x.sibling$
        }
        \Else(\comment*[h]{if $x.sibling.left$ is black}\explain*[f]{case 4 $|$ $x.sibling.right$ for case 8}){
          color $x.sibling$ like $x.parent$\\
          color $x.parent$ \textbf{and} $x.sibling.right$ black\explain*{$x.sibling.left$ for case 8}
          rotate $x.parent$ left\explain*{rotate right for case 8}
          \textbf{break} with $x \gets T.root$
        }
      }
    }
  }

  \comment{fix II and IV}
  color $x$ black
\end{algorithm}


\subsection{AVL Trees}
\subsubsection{Property}
For every node, the heights of the left and right subtrees differ by at most 1.

\subsubsection{Insertion}
Insert the new node as in a normal binary search tree and then rebalance the tree by performing rotations from the inserted node to the root.

\subsubsection{Deletion}
Delete the node as in a normal binary search tree and then rebalance the tree by performing rotations from the deleted node to the root.

\subsection{Hashing}
\subsubsection{Chaining}
Chained hashing uses a linked list to store all elements that hash to the same value. The time complexity of chained hashing is $\bigO(1+\alpha)$, where $\alpha$ is the load factor.

\subsubsection{Open Addressing}
Open addressing uses the next free slot in the hash table to store elements that hash to the same value. The time complexity of open addressing is $\bigO(1+\alpha)$, where $\alpha$ is the load factor.


\section{Graphs}
\subsection{Topological Sort}
\begin{alg}
  \signed{topologicalSort}{G}{L}
  \params{A graph $G$}{A topological sorted list of vertices $L$}

  $L \gets \emptyset$\;
  \ForEach{vertex $u$ of $G$}{
    \lIf{$u$ is not visited}{
      \texttt{dfs($u, L$)}
    }
  }
  \Return $L$\;
\end{alg}

\begin{alg}
  \signed{topoDFS}{u, L}{\empty}
  \params{A vertex $u$ and a list $L$}{\empty}

  mark $u$ as found\;
  \ForEach{vertex $v$ of $u$'s neighbors}{
    \lIf{$v$ is not visited}{
      \texttt{dfs($v$)}
    }
    \lElseIf{$v$ is found}{
      \textbf{error} "cycle detected"
    }
  }
  mark $u$ as visited\;
  $L \gets \{u\} \cup L$
\end{alg}

\subsection{Strongly Connected Components: Kosaraju-Sharir's Algorithm}
\begin{alg}
  \signed{findSCCs}{G}{SCCs}
  \params{A graph $G$}{A list of strongly connected components $SCCs$}
  \BlankLine

  $L \gets \emptyset$\;
  \ForEach{vertex $u$ of $G$}{
    \lIf{$u$ is not visited}{\texttt{dfs($G, u, L$)}}
  }

  \BlankLine
  invert edges of $G$ and reset visited flags\;
  \BlankLine

  $SCCs \gets \emptyset$\;
  \ForEach{vertex $u\in L$ in reverse order}{
    \lIf{$u$ is not visited}{
      $scc \gets \emptyset$\;
      \texttt{dfs($G, u, scc$)}
      $SCCs \gets SCCs \cup \{scc\}$
    }
  }
  \Return $SCCs$\;
\end{alg}

\begin{alg}
  \signed{dfs}{G, u, L}{\empty}
  \params{A graph $G$, a vertex $u$, and a list $L$}{\empty}
  \BlankLine

  mark $u$ as found\;
  \ForEach{vertex $v$ of $u$'s neighbors}{
    \lIf{$v$ is not found}{\texttt{dfs($v$)}}
  }
  mark $u$ as visited\;
  $L \gets L \cup \{u\}$
\end{alg}

\subsection{Minimal Spanning Trees}
\subsubsection{Kruskal}
\begin{algorithm}[H]
  \caption{\texttt{kruskal(G)}}

  \KwData{A graph $G$}
  \KwResult{A minimal spanning tree $T$}
  \BlankLine

  $T \gets \emptyset$\;
  \lForEach(\explain*[f]{interpret $V$ as forest of trees, each connected by $E$}){vertex $v$ of $G$}{
    \texttt{makeSet($v$)}
  }
  sort edges of $G$ by weight\;
  \ForEach{edge $(u, v)$ of $G$}{
    \If{\texttt{findSet($u$)} $\neq$ \texttt{findSet($v$)}}{
      $T \gets T \cup \{(u, v)\}$\;
      \texttt{union($u, v$)}
    }
  }
  \Return $T$\;
\end{algorithm}

\subsubsection{Prim}
\begin{alg}
  \signed{prim}{G, w, r}{T}
  \params{A graph $G$ with weights $w$, and a root vertex $r$}{A minimal spanning tree $T$}
  \BlankLine

  $T \gets \emptyset$\;
  $Q \gets G.V$ with $\infty$ priorities\;
  $r.priority \gets 0$\;
  \While{$Q$ is not empty}{
    $u \gets \texttt{extractMin(Q)}$\;
    \ForEach{vertex $v$ of $u$'s neighbors}{
      \If{$v \in Q$ \textbf{and} $w(u, v) < v.priority$}{
        $v.priority \gets w(u, v)$\;
        $T \gets T \cup \{(u, v)\}$
      }
    }
  }
  \Return $T$\;
\end{alg}

\subsection{Find Bridges}
The \algrefp{findBridges} algorithm finds all bridges in a graph. A bridge is an edge whose removal increases the number of connected components in the graph. The algorithm has a time complexity of $\bigO(E(V+E))=\bigO(E\cdot V+E^2)$.\par
\begin{alg}
  \signed{findBridges}{G}{B}
  \params{A graph $G$}{A list of bridges $B$}

  $B \gets \emptyset$\;
  $s \gets G.V.first$\;
  \ForEach{edge $e\in G.E$}{
    $ICC \gets$ \texttt{findICC($G\setminus \{e\}, s$)}\explain*{an initial connected component describes the vertices reachable from $s$}
    \lIf{$|ICC.V| < |G.V|$}{$B \gets B \cup \{e\}$}
  }
  \Return $B$\;
\end{alg}

The DFS algorithm \algrefp{findICC} finds an initial connected component (ICC) of a graph $G$ starting from a vertex $u$. The algorithm (like other DFSs) has a time complexity of $\bigO(V+E)$.\par
\begin{alg}
  \signed{findICC}{G, u}{ICC}
  \params{A graph $G$, a vertex $u$}{An initial connected component $ICC$}

  mark $u$ as visited\;
  $ICC \gets \{u\}$\;
  \ForEach{neighbor $v$ of $u$}{
    \lIf{$v$ is not visited}{
      $ICC \gets ICC \cup \texttt{findICC(G, v)}$
    }
  }
  \Return $ICC$\;
\end{alg}

\subsection{Shortest Paths}
\begin{alg}
  \signed{initializeSingleSource}{G, s}{\empty}
  \params{A graph $G$ and a source vertex $s$}{\empty}
  \BlankLine

  \ForEach{vertex $v\in G.V$}{
    $v.distance \gets \infty$\;
    $v.parent \gets nil$\;
  }
  $s.distance \gets 0$
\end{alg}

\begin{alg}
  \signed{relax}{u, v, w}{\empty}
  \params{Two vertices $u$ and $v$, and a weight $w$}{\empty}
  \BlankLine

  \If{$v.distance > u.distance + w(u, v)$}{
    $v.distance \gets u.distance + w(u, v)$\;
    $v.parent \gets u$
  }
\end{alg}

\subsubsection{Bellman-Ford}
The \algrefp{bellmanFord} finds the shortest path from a source vertex $s$ to all other vertices in a graph $G$. The algorithm has a time complexity of $\bigO(V\cdot E)$.\par
\begin{alg}
  \signed{bellmanFord}{G, w, s}{\empty}
  \params{A graph $G$ with weights $w$, and a source vertex $s$}{A shortest path by traversing the parent pointers from a vertex $u$}
  \BlankLine

  \texttt{initializeSingleSource($G, r$)}\;
  \ForEach{vertex in $G$}{
    \lForEach{edge $(u, v)\in G.E$}{\texttt{relax($u, v, w$)}}
  }
  \ForEach{edge $(u, v)\in G.E$}{
    \lIf{$v.distance > u.distance + w(u, v)$}{\textbf{error} "negative cycle detected"}
  }
\end{alg}

\subsubsection{Dijkstra}
The \algrefp{dijkstra} algorithm finds the shortest path from a source vertex $s$ to all other vertices in a graph $G$. The algorithm has a time complexity of $\bigO(E\log V)$ when the Queue is implemented with a Min-Heap. Using Fibonacci-Heaps the time complexity can be reduced to $\bigO(E+V\log V)$.\par
\begin{algorithm}[H]
  \caption{\texttt{dijkstra($G, w, s$)}}\label{alg:dijkstra}

  \KwData{A graph $G$ with weights $w$, and a source vertex $s$}
  \KwResult{A shortest path by traversing the parent pointers from a vertex $u$}
  \BlankLine

  \texttt{initializeSingleSource($G, s$)}\;
  $S \gets \emptyset$\explain*{$S$ is a set of vertices with known shortest paths}
  $Q \gets G.V$\explain*{$Q$ is a (min-)priority queue by vertex distance}

  \While{$Q$ is not empty}{
    $u \gets \texttt{extractMin(Q)}$\;
    $S \gets S \cup \{u\}$\;
    \lForEach(\explain*[f]{position in $Q$ is implicitly updated}){vertex $v$ of $u$'s neighbors}{\texttt{relax($u, v, w$)}}
  }
\end{algorithm}

The \algref{dijkstraSinglePair} finds the shortest path from a source vertex $s$ to a target vertex $t$ in a graph $G$. The algorithm has a time complexity of $\bigO(V\log V+E)+\bigO(V)=\bigO(V\log V+E)$ (based on the single source \texttt{dijsktra}).\par
\begin{alg}
  \signed{dijkstraSinglePair}{G, w, s}{path}
  \params{A graph $G$ with weights $w$, and source and target vertices $s$ and $t$}{A shortest $path$ from $s$ to $t$}
  \BlankLine

  \texttt{dijkstra($G, w, s$)}\;
  $path \gets \{t\}$\;
  \While{$t \neq s$}{
    $path \gets \{t.parent\} \cup path$\;
    $t \gets t.parent$
  }
  \Return $path$
\end{alg}

\subsubsection{Floyd-Warshall}
The \algref{floydWarshall} algorithm finds the shortest path between all pairs of vertices in a graph $G$. The algorithm has a time complexity of $\bigO(V^3)$.\par
\begin{alg}
  \signed{floydWarshall}{W}{D}
  \params{Weighted adjacency matrix $W$ of a graph $G$}{Matrix of shortest distances $D$}

  $n \gets \text{number of vertices in } G$\;
  $D \gets W$\;

  \For{$k \gets 1$ \KwTo $n$}{
    \For{$i \gets 1$ \KwTo $n$}{
      \For{$j \gets 1$ \KwTo $n$}{
        $D[i,j] \gets \min(D[i,j], D[i,k] + D[k,j])$
      }
    }
  }

  \Return $D$\;
\end{alg}



\section{Hard problems with Smart Solutions}
\subsection{Dynamic Programming}
Dynamic programming is a method for solving complex problems by breaking them down into simpler subproblems. It is applicable to problems exhibiting the property of overlapping subproblems and optimal substructure.

The four steps to solve a problem using dynamic programming are:
\begin{enumerate}
  \item[1.] Characterize the structure of an optimal solution.
  \item[2.] Recursively define the value of an optimal solution.
  \item[3.] Compute the value of an optimal solution in a bottom-up fashion.
  \item[4.] Construct an optimal solution from computed information.
\end{enumerate}

\subsection{NP-Completeness}
A decision problem is NP-complete if it is in NP and every problem in NP can be reduced to it in polynomial time. The most famous NP-complete problem is the Boolean satisfiability problem (SAT).

To prove NP-completeness, one can use the following steps:
\begin{enumerate}
  \item Prove that the problem is in NP.
  \item Choose a known NP-complete problem and reduce it to the problem at hand.
\end{enumerate}

\subsection{Solving NP-Complete Problems}
\begin{enumerate}
  \item[1.] \textbf{Approximation Algorithms:} Solution close to optimal by a ratio $\rho(n) = \frac{\text{approximate solution}}{\text{optimal solution}}$. Greedy algorithms, etc.
  \item[2.] \textbf{Heuristic Algorithms:} No guarantee, but resonably good solutions usually. Local search, genetic algorithms, etc.
  \item[3.] \textbf{Exact Algorithms:} Find the optimal solution. Backtracking, branch and bound, etc.
\end{enumerate}

\end{document}