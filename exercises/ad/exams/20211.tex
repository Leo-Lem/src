\documentclass{article}

\usepackage{exercises}
\withsolutions

\begin{document}
\begin{exercises}{Big-O Notation}
\item Sei $f(n) = 23 \cdot n^{\log_2{9}}$. Nennen Sie zwei Funktionen $g_1(n)$ und $g_2(n)$, so dass die folgenden Eigenschaften erfüllt sind:
\begin{itemize}
  \item $g_1$ und $g_2$ wachsen asymptotisch echt schneller als $f$
  \item $g_1$ und $g_2$ wachsen nicht polynomiell schneller als $f$
  \item $g_1$ und $g_2$ wachsen asymptotisch nicht gleich schnell
\end{itemize}
Begründen Sie kurz, warum Ihre gewählten Funktionen die geforderten Eigenschaften erfüllen.
\item Entscheiden Sie, ob die folgenden Aussagen über Funktionen der Form $f: \mathbb{N} \to \mathbb{R}$ wahr oder falsch sind und beweisen bzw. widerlegen Sie diese. Falls Sie bekannte Rechenregeln aus der Vorlesung benutzen, so nennen Sie diese beim Namen bzw. referenzieren Sie diese.
\begin{itemize}
  \item $\log n = \bigO(\sqrt{n})$
  \item $\sum_{i=1}^n 3i - 7 = \Theta(n^2)$
  \item $\frac{n^2\cdot 2\sqrt{n})^3}{n^3-3} = \Omega(n) $
  \item $\prod_{i=1}^n (i + n) = \bigO(n^{2n})$
\end{itemize}
\end{exercises}

\begin{exercises}{Master-Theorem}
\item Entscheiden Sie, ob die folgende Rekursionsgleichung mithilfe des Mastertheorems lösbar ist oder nicht. Ist das Mastertheorem anwendbar, so lösen Sie die Rekursionsgleichung damit. Ist das Mastertheorem nicht anwendbar, so lösen Sie die Rekursionsgleichung mit der Substitutionsmethode.
\begin{equation*}
  T(n) = \begin{cases}
    2\cdot T(n/8)+\sqrt{n} & \text{für } n > 1 \\
    17                     & \text{sonst}
  \end{cases}
\end{equation*}
\item Für die Laufzeit $T(n)$ eines Algorithmus $A$ gelte die Rekursion
\begin{equation*}
  T(n) \leq \begin{cases}
    \max_{1\leq k\leq n}(3T(k-1)+T(n-k))+2\cdot n^2 & \text{für } n > 0 \\
    1                                               & \text{sonst}
  \end{cases}
\end{equation*}
Bestimmen und beweisen Sie eine möglichst exakte obere Schranke (im $\bigO$-Kalkül) für die Laufzeit von Algorithmus $A$.
\hint{Raten Sie eine Lösung \& verifizieren Sie diese dann durch vollständige Induktion.}
\end{exercises}

\begin{exercise}{Mergesort}
  Sortieren Sie das Array $A = [13,11,5, -1,3,0,42,1]$ durch Anwendung des Algorithmus MERGESORT aus der Vorlesung. Zur Erinnerung finden Sie Teile des zugehörigen Pseudocodes in Algorithmus 1. Geben Sie den Inhalt des Arrays $A$ nach jeder Durchführung der Merge-Operation in Zeile 5 an.
  \begin{algorithm}[ht]
    \caption{MERGESORT($A, l, r$)}
    \KwData{$A = [A[1], \ldots, A[n]]$}
    \KwResult{$A$ sortiert}
    \If{$l < r$}{
      $p \gets \lfloor (l + r)/2 \rfloor$ \\
      MERGESORT($A, l, p$) \\
      MERGESORT($A, p + 1, r$) \\
      MERGE($A, l, p, r$)
    }
  \end{algorithm}
\end{exercise}

\begin{eexercises}{Algorithmenentwurf}{
    Für ein Array $A[1, \ldots, n]$ von $n \in \mathbb{N}$ ganzen Zahlen nennen wir ein Index-Paar $(i, j)$ verdreht, wenn $i < j$ und $A[i] < A[j]$.
  }
  \item Geben Sie ein Array $A$ bestehend aus vier Zahlen und der kleinstmöglichen Anzahl an verdrehten Index-Paaren an. Um wieviele verdrehte Index-Paare handelt es sich genau?
  \item Geben Sie ein Array $A$ bestehend aus vier Zahlen und der größtmöglichen Anzahl an verdrehten Index-Paaren an. Um wieviele verdrehte Index-Paare handelt es sich genau?
  \item Geben Sie ein Array $A$ bestehend aus vier Zahlen und exakt drei verdrehten Index-Paaren an.
  \item Entwerfen Sie einen rekursiven Algorithmus, der bei Eingabe eines Array $A$ von $n$ ganzen Zahlen in Zeit $\mathcal{O}(n \log n)$ die Anzahl an verdrehten Index-Paaren bestimmt. Sie dürfen Pseudocode angeben, müssen es aber nicht. Sollte Ihr Algorithmus auf bekannten Algorithmen aus der Vorlesung basieren, reicht es die notwendigen Änderungen (klar \& deutlich!) zu beschreiben.
  \item Beweisen Sie, dass Ihr Algorithmus die geforderte Laufzeitschranke einhält.
\end{eexercises}

\begin{exercises}{Suchbäume}
\item Wählen Sie sechs unterschiedliche Zahlen mit Werten zwischen 0 und 50. In welcher Reihenfolge müssen Ihre gewählten Zahlen in einen (anfangs leeren) binären Suchbaum eingefügt werden, damit dieser maximal unbalanciert ist? Zeichnen Sie den resultierenden binären Suchbaum.
\item Wählen Sie sechs (neue) unterschiedliche Zahlen mit Werten zwischen 0 und 50. In welcher Reihenfolge müssen Ihre gewählten Zahlen in einen (anfangs leeren) binären Suchbaum eingefügt werden, damit dieser minimal unbalanciert ist? Zeichnen Sie den resultierenden binären Suchbaum.
\item Gegeben sei ein AVL-Baum für die Schlüssel $\{17,19,21,23,42,45,47,53,66\}$. Der Zustand der Datenstruktur ist in Abbildung 1 dargestellt. Löschen Sie das Element mit Schlüssel 42 gemäß der Löschoperation für AVL-Bäume aus der Vorlesung. Geben Sie dabei den Zustand der Datenstruktur nach jedem Entfernen eines Knotens und nach jeder Rotation an.
% TODO: insert avl tree
\end{exercises}

\begin{eexercises}{Dynamische Programmierung}{
    Betrachten Sie ein eindimensionales Gitter aus $n$ Feldern nummeriert von 1 bis $n$. Die kleine Ameise Esiema startet auf Feld 1. Befindet sich Esiema auf einem Feld am Rand (Feld 1 oder $n$), so bewegt sie sich im nächsten Schritt mit Wahrscheinlichkeit 1 in die freie Richtung (weg vom Rand). Befindet sich Esiema auf einem anderen Feld, so bewegt sie sich im nächsten Schritt mit Wahrscheinlichkeit 1/2 um ein Feld nach links und mit Wahrscheinlichkeit 1/2 um ein Feld nach rechts.
  }
  \item Abbildung 2 zeigt einen Ausschnitt der Tabelle mit den Wahrscheinlichkeiten, dass Esiema nach $i$ Schritten auf Feld $x$ ist. Ergänzen Sie die sechs fehlenden Werte für $x \in \{1,2,\ldots,6\}$ und $i = 5$. Sie können die fehlenden Werte direkt in die Tabelle eintragen oder auf eigenes Papier schreiben.
  \begin{tabular}{c|cccc ccc}
    $i \backslash x$ & 1      & 2      & 3      & 4      & 5      & 6      & \ldots \\
    \hline
    0                & 1      & 0      & 0      & 0      & 0      & 0      & \ldots \\
    1                & 0      & 1      & 0      & 0      & 0      & 0      & \ldots \\
    2                & 1/2    & 0      & 1/2    & 0      & 0      & 0      & \ldots \\
    3                & 0      & 3/4    & 0      & 1/4    & 0      & 0      & \ldots \\
    4                & 3/8    & 0      & 1/2    & 0      & 1/8    & 0      & \ldots \\
    5                &        &        &        &        &        &        & \ldots \\
    \ldots           & \ldots & \ldots & \ldots & \ldots & \ldots & \ldots & \ldots \\
  \end{tabular}
  \item Betrachten Sie ein zweidimensionales Array (Tabelle) $D$ der Größe $(n + 1) \times n$. Ein Eintrag $D[i,x]$ für $i \in \{0,1,\ldots,n\}$ und $x \in \{1,2,\ldots,n\}$ entspricht dabei der Wahrscheinlichkeit, dass Esiema nach $i$ Schritten auf Feld $x$ ist. Entwerfen Sie eine Rekursion zum Berechnen von $D$ und begründen Sie kurz (!) in Worten die Korrektheit der Rekursion.
  \item Geben Sie einen Algorithmus in Pseudocode an, der mittels dynamischer Programmierung bestimmt, auf welchen Feldern sich Esiema nach $n$ Schritten mit der größten Wahrscheinlichkeit aufhält.
\end{eexercises}

\begin{eexercises}{Algorithmenentwurf II}{
    Nach Ihrem Informatik-Studium starten Sie als Comedian durch. Sie sind sehr erfolgreich und haben bald mehr Anfragen als Sie annehmen können. Jede Anfrage besteht aus einer angebotenen Gage (in Euro) und einer Deadline (in Tagen), bis zu der der Auftritt erfolgen muss. Aus zeitlichen Gründen können Sie pro Tag maximal einen Auftritt absolvieren. Sie wollen eine Teilmenge von Anfragen annehmen, die Ihre Einnahmen maximiert. Sie erinnern sich an Ihr Informatik-Studium und entwerfen einen gierigen Algorithmus, der dieses Problem optimal in polynomieller Zeit löst. Als Eingabe erhält Ihr Algorithmus zwei Arrays $G$ und $D$ mit jeweils $n \in \mathbb{N}$ Elementen. Der Eintrag $G[i] > 0$ entspricht der Gage die Sie bei der $i$-ten Anfrage erhalten. Der Eintrag $D[i] > 0$ entspricht der Deadline der $i$-ten Anfrage.
    \hint{zB. seien ihre aktuellen Anfragen gegeben durch 1. Gage: 100€, Deadline: 2 Tage 2. Gage: 200€, Deadline: 2 Tage 3. Gage: 300€, Deadline: 1 Tag 4. Gage: 50€, Deadline: 1 Tag. Eine optimale Auswahl an Anfragen wären Anfragen 2 und 3 für insgesamt 500€.}
  }
  \item Welches Angebot sollte Ihr Algorithmus zuerst akzeptieren und an welchem Tag sollte der zugehörige Auftritt stattfinden? Welches Angebot sollte Ihr Algorithmus als zweites akzeptieren und an welchem Tag sollte der zugehörige Auftritt stattfinden? Begründen Sie diese Wahlen kurz (!) in Worten.
  \item Beschreiben Sie die allgemeine Vorgehensweise des gierigen Algorithmus kurz in Worten.
  \item Geben Sie Ihren Algorithmus in Pseudocode an. Der Algorithmus soll die Indizes der auszuwählenden Anfragen sowie die dadurch erzielten Einnahmen ausgeben.
\end{eexercises}
\end{document}