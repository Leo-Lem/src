\documentclass{article}

\usepackage[solutions]{xrcise}

\begin{document}
\sheet[2012]{Ersttermin (Rarey)}

\begin{exercise}{Multiple Choice}
  Bei allen Multiple-Choice Fragen ist genau ein Kreuz zu setzen. Sollte mehr als ein Kreuz gesetzt sein, wird die Teilaufgabe mit 0 Punkten gewertet.
  \begin{enumerate}
    \item Gegeben sind die Funktionen \[
            n^2 \quad n! \quad n^{2n} \quad \sqrt{3} \quad \log n
          \] Welche der folgenden Aussagen gilt?
          \begin{itemize}
            \item[$\square$] Alle Funktionen sind polynomiell.
            \item[$\square$] $n!$ ist die am stärksten wachsende Funktion
            \item[$\square$] $\sqrt{3} n = \bigO(\log(n))$
            \item[$\square$] $n^2 + o(n^2) = \Theta(n^2)$
          \end{itemize}
    \item Seien $f(n), g(n)$ Funktionen. Weiter sei $h$ definiert durch $h(n) = \max(f(n), g(n))$. Gilt dann
          \begin{itemize}
            \item[$\square$] $h(n) = \bigO(f(n))$
            \item[$\square$] $h(n) = \bigO(f(n) + g(n))$
            \item[$\square$] $h(n) = \omega(f(n) + g(n))$
            \item[$\square$] $h(n) = \bigO(f(n) - g(n))$
          \end{itemize}
    \item Beschreiben Sie die Datenstruktur Schlange, die sich durch die folgende Sequenz von Operationen ergibt:\par
          \begin{quote}\textalgo{enqueue(15), rear(), enqueue(7), enqueue(8), dequeue(), enqueue(9), dequeue(), head(), enqueue(10)}\end{quote}
    \item Wir betrachten einen Baum $T$ mit Höhe $h$ (die Wurzel hat die Höhe 0) und $n$ Knoten. Die inneren Knoten von $T$ haben maximal drei Kinder. Welche der folgenden Aussagen ist wahr?
          \begin{itemize}
            \item[$\square$] $h$ ist immer $\log_3(n)$
            \item[$\square$] $T$ hat mindestens $n$ Kanten
            \item[$\square$] $T$ hat immer mehr Blätter als innere Knoten
            \item[$\square$] $T$ hat mindestens $n - 3h$ innere Knoten.
          \end{itemize}
    \item Sei $G = (V,E)$ ein zusammenhängender, ungerichteter Graph mit der Kanten- gewichtsfunktion $w$. Angenommen $w(e) = c, c$ konstant, für alle Kanten, welche Laufzeit lässt sich dann zur Berechnung des minimalen Spannbaumes in $G$ erreichen?
          \begin{itemize}
            \item[$\square$] $\bigO(|V| \log |E|)$
            \item[$\square$] $\bigO(|V| + |E|)$
            \item[$\square$] $\bigO(|E| \log |V|)$
            \item[$\square$] $\bigO(|V|^2)$
          \end{itemize}
    \item Fügen Sie den Wert 12 in die folgende Hashtabelle mit offener Addressierung und linearer Sondierung ein. Als Hashfunktion wird $h(x) = x \mod 7$ benutzt.
          \begin{center}
            \begin{tabular}{|c|c|c|c|c|c|c|c|}
              \hline
              0 & 1 & 2 & 3 & 4 & 5 & 6  \\
              \hline
              \hline
                & 5 & 9 &   &   & 5 & 13 \\
              \hline
            \end{tabular}
          \end{center}
    \item Führen Sie eine Linksrotation auf den Knoten $a$ aus.\begin{figure}
  \centering

  \caption{Tree before rotation}\label{fig:treerotate}
\end{figure}
    \item Seien $L, L' \in P$. Welche der folgenden Aussagen ist falsch?
          \begin{itemize}
            \item[$\square$] Wenn $L \in NP$ gilt, dann folgt $P = NP$
            \item[$\square$] $\forall L'' \in NPC:L \leq_p L''$
            \item[$\square$] $L \cap L' \in P$
            \item[$\square$] $L \cup L' \in P$
          \end{itemize}
    \item Welches ist die untere Schranke für die Anzahl der Vergleichsoperationen bei vergleichsbasierten Sortieralgorithmen?
          \begin{itemize}
            \item[$\square$] $\Omega(n)$
            \item[$\square$] $\Omega(\log(n))$
            \item[$\square$] $\Omega(n \log(n))$
            \item[$\square$] $\Omega(n^2)$
          \end{itemize}
    \item In einem minimalen Spannbaum eines Graphen mit $n$ Knoten
          \begin{itemize}
            \item[$\square$] kann die Kante mit dem höchsten Gewicht enthalten sein
            \item[$\square$] darf die Kante mit dem kleinsten Gewicht nicht enthalten sein
            \item[$\square$] können mehr als $n - 1$ Kanten enthalten sein
            \item[$\square$] muss die Summe der Kantengewichte durch 2 teilbar sein
          \end{itemize}
    \item In einem Rot-Schwarz-Baum gilt (ohne Betrachtung der Wächter) für alle roten Knoten $r$
          \begin{itemize}
            \item[$\square$] der Vor-Vorgänger von $r$ ist rot, falls der Elter von $r$ nicht die Wurzel ist
            \item[$\square$] der Bruder Knoten von $r$ ist rot
            \item[$\square$] Wenn $r$ kein Blatt ist, dann hat $r$ zwei Kinder
            \item[$\square$] der Onkel von $r$ ist schwarz
          \end{itemize}
  \end{enumerate}
  \begin{solution}

  \end{solution}
\end{exercise}

\begin{exercise}{Algorithmenanalyse}
  \begin{enumerate}
    \item Betrachten Sie die folgenden Code-Segmente. Geben Sie für jedes Segment eine möglichst dichte Schranke für die asymptotische Laufzeit im O-Kalkül in Abhängigkeit von $N$, bzw. $N = \text{length}[A]$ an. Bei rekursiven Algorithmen formulieren Sie zuvor die Rekurrenzgleichung. (Teil-Arrays werden immer als Kopie übergeben, das Kopieren kostet keine Zeit.)
          \begin{enumerate}
            \item $\bigO(?)$
                  \begin{alg}\signed{alg1}{}{\empty}
                    \For{$i \gets 0$ \KwTo $N$}{
                      \For{$j \gets N$ \textbf{downto} $1$}{
                        \textbf{do} $sum \gets sum + 1$\;
                      }
                    }
                  \end{alg}
            \item $\bigO(?)$
                  \begin{alg}\signed{alg2}{}{\empty}
                    \For{$i \gets 1$ \KwTo N}{
                      \textbf{do} $j \gets N$\;
                      \While{j > 1}{
                        \textbf{do} $sum \gets sum + j$\;
                        \textbf{do} $j \gets j/2$\;
                      }
                    }
                  \end{alg}
            \item $\bigO(?)$
                  \begin{alg}\signed{alg3}{}{\empty}
                    \While{$N > 1$}{
                      \For{$j \gets 1$ \KwTo $N$}{
                        \textbf{do} $sum \gets sum + j$\;
                      }
                      $N \gets N/2$\;
                    }
                  \end{alg}
            \item $T(n)=\{?$ and $\bigO(?)$
                  \begin{alg}\signed{alg4}{A}{\empty}
                    $middle \gets \text{length}[A]/2$\;
                    $sum \gets sum + A[1]$\;
                    \If{$\text{length}[A] > 2$}{
                      \call{alg4}{A[1 \KwTo middle]}\;
                      \call{alg4}{A[middle + 1 \KwTo \text{length}[A]]}\;
                    }
                  \end{alg}
            \item $T(n)=\{?$ and $\bigO(?)$
                  \begin{alg}\signed{alg5}{A}{\empty}
                    \For{$i \gets 1$ \KwTo $\text{length}[A]$}{
                      \textbf{do} $sum \gets sum + A[i]^2$\;
                    }
                    \If{$\text{length}[A] > 3$}{
                      \call{alg5}{A[1 \KwTo \text{length}[A]/3]}\;
                      \call{alg5}{A[\text{length}[A]/3 \KwTo 2 \cdot \text{length}[A]/3]}\;
                      \call{alg5}{A[2 \cdot \text{length}[A]/3 \KwTo \text{length}[A]]}\;
                    }
                  \end{alg}
            \item $T(n)=\{?$ and $\bigO(?)$
                  \begin{alg}\signed{alg6}{A}{\empty}
                    \For{$i \gets 1$ \KwTo $\text{length}[A]$}{
                      \textbf{do} $sum \gets sum + A[i]$\;
                    }
                    \If{$\text{length}[A] > 1$}{
                      \call{alg6}{A[1 \KwTo \text{length}[A] - 2]}\;
                    }
                  \end{alg}
            \item $T(n)=\{?$ and $\bigO(?)$
                  \begin{alg}\signed{alg7}{A}{\empty}
                    \textbf{do} $sum \gets sum + 1$\;
                    \If{$\text{length}[A] > 1$}{
                      $\call{alg7}{A[1 \KwTo \text{length}[A] - 2]}$\;
                      $\call{alg7}{A[2 \KwTo \text{length}[A] - 1]}$\;
                    }
                  \end{alg}
          \end{enumerate}
    \item Bestimmen Sie die asymptotische Laufzeit des folgenden Algorithmus mit der Substitutionsmethode, geben Sie Ihren Lösungsweg an:
          \begin{alg}\signed{alg8}{A, N, n}{\empty}
            \If{$n \geq 2$}{
              \For{$i \gets 1$ \KwTo $n$}{
                \textbf{do} $\text{swap}(A[i], A[r])$\;
                \call{alg8}{A, N, n - 1}\;
                \textbf{do} $\text{swap}(A[r], A[i])$\;
              }
            }
            \Else{
              $sum \gets 0$\;
              \For{$i \gets 1$ \KwTo $N$}{
                \textbf{do} $s \gets s + A[i]$\;
              }
            }
          \end{alg}
          Die Funktion wird aufgerufen mit: $\text{alg8}(A, \text{length}[A], \text{length}[A])$
          \hint{Verwenden Sie folgende Notation: $[n]_i = n \cdot (n - 1) \cdot \ldots \cdot (n - i)$}
          \hint{Verwenden Sie die Abschätzung $[n]_i \leq n!$, wenn dies nicht zu einer Erhöhung der Laufzeitschranke im O-Kalkül führt.}
  \end{enumerate}

  \begin{solution}

  \end{solution}
\end{exercise}

\begin{exercise}{Quicksort}
  \begin{enumerate}
    \item
          Gegeben sei die bekannte Partitionsfunktion aus dem Quicksort-Algorithmus
          \begin{alg}
  \signed{Partition}{A, l, r}{\empty}
  $i \gets l - 1$\;
  \For{$j \gets l$ \KwTo $r - 1$}{
    \If{$A[j] \leq A[r]$}{
      $i \gets i + 1$\;
      $\text{SWAP}(A[i], A[j])$\;
    }
  }
  $\text{SWAP}(A[i + 1], A[r])$\;
  \KwRet{$(i + 1)$}\;
\end{alg}
          Dabei gehen wir in dieser Aufgabe von Arrays mit 0-basiertem Index aus. Sie befinden sich in Zeile 4 der Partition-Funktion mit $l = 0, r = 6$ und $j = 3, i = 0$. Werten Sie alle Schritte bis zum Erreichen der Zeile 8 aus. Protokollieren Sie den Speicherzustand jeweils beim Erreichen der Zeile 4 und der Zeile 8.
          \begin{itemize}
            \item[Zeile 4] $A: 3, 8, 6, 4, 6, 4, 5$ \hfill $j = 3, i = 0$
            \item[Zeile 4] $A: ?$ \hfill $j = 4, i = ?$
            \item[Zeile 4] $A: ?$ \hfill $j = 5, i = ?$
            \item[Zeile 8] $A: ?$ \hfill $j = 5, i = ?$
          \end{itemize}
    \item Ist Quicksort mit der obigen Partition Funktion ein stabiles Sortierverfahren? Begründen Sie Ihre Antwort und zeigen Sie entweder, dass Partition stabil ist oder schlagen Sie eine Modifikation von Quicksort vor, welche diesen stabilisiert.
    \item Gegeben sei ein Array der Länge $N$, welches ausschließlich Elemente aus der Menge $0,1,2$ enthält. Entwerfen Sie einen Algorithmus, der die Elemente in $\bigO(n)$ Zeit sortiert. Der Algorithmus soll vergleichs-/austauschbasiert und in-place arbeiten, was bedeutet, dass er maximal $\bigO(1)$ zusätzlichen Speicher nutzen darf. Skizzieren Sie Ihre Idee mit Pseudo-Code und einem beschreibenden Text.
  \end{enumerate}

  \begin{solution}

  \end{solution}
\end{exercise}

\begin{exercise}{Minimaler Spannbaum}
  \begin{figure}
  \centering

  \caption{A graph}\label{fig:mstgraphrarey}
\end{figure}
  \begin{enumerate}
    \item Zeichnen Sie in den oben angegebenen Graphen einen minimalen Spannbaum ein. Ist dieser Spannbaum eindeutig? Begründen Sie Ihre Antwort.
    \item Gegeben sei ein Graph $G = (V,E)$. Entwerfen Sie einen rekursiven Algorithmus \texttt{Label(V, E, u, ’label’)}, der einen Knoten $u$ und alle von $u$ aus erreichbaren Knoten mit der Markierung ’label’ versieht. Der Algorithmus soll die Laufzeit $\bigO(|V| + |E|)$ haben. Beschreiben Sie Ihren Algorithmus und geben Sie Pseudo-Code für die Funktion \texttt{Label} an.
    \item In einem Graphen $G = (V,E)$ mit Kantengewichten bezeichnen wir eine Kante $e$ eines minimalen Spannbaums $T$ als kritisch, wenn das Entfernen von $e$ das Gewicht des minimalen Spannbaums in $G$ vergrößert. Entwickeln Sie einen Algorithmus, der in $\bigO(|V| + |E|)$ Zeit entscheidet, ob $e$ kritisch ist. Nutzen Sie dabei die Funktion \texttt{Label} aus der vorausgehenden Aufgabe als Hilfsfunktion. Beschreiben Sie kurz Ihren Lösungsweg und geben Sie den Algorithmus im Pseudo-Code an. Argumentieren Sie außerdem in wenigen Stichpunkten, warum Ihr Algorithmus korrekt ist.
  \end{enumerate}

  \begin{solution}

  \end{solution}
\end{exercise}

\begin{exercise}{Graphen}
  \begin{figure}
  \centering

  \caption{Train line graph.}\label{fig:trains}
\end{figure}
  Eine neue Eisenbahnlinie soll gebaut werden und Sie wurden mit ihrer Planung beauftragt. Zur Verfügung steht Ihnen eine topographische Landkarte mit möglichen Streckenverlaufspunkten $p_i$. Auf der Karte sind die Distanzen in km zwischen den Punkten eingetragen sowie die Höhe der Streckenverlaufspunkte über Normalnull in Metern. Ihr Auftrag lautet, eine Bahnlinie zwischen zwei gegebenen Punkten $s, t$ zu bestimmen, auf welcher ein Zug möglichst wenig Energie umsetzt. Die umgesetzte Energie bestimmt sich dabei wie folgt:
  \begin{itemize}
    \item 1 Energieeinheit pro km Strecke
    \item 2 Energieeinheiten pro Anstieg um einen Meter
    \item -1 Energieeinheit pro Abfall um einen Meter
  \end{itemize}
  \begin{enumerate}
    \item Wieviel Energie wird auf dem in der Karte grau markierten Pfad von $s$ nach $t$ (über $p_0$ und $p_3$) umgesetzt? Gibt es einen energetisch günstigeren Pfad?
    \item Modellieren Sie das Problem so, dass Sie einen aus der Vorlesung bekannten Algorithmus benutzen können. Beschreiben Sie Ihre Modellierung und nennen Sie den Algorithmus, der angewendet werden kann.
    \item Durch Einsatz einer moderneren Lok ändert sich der Energieverbrauch pro Anstieg um einen Meter auf 1 Energieeinheit. Zeigen Sie, dass nun im allgemeinen Fall für zwei beliebige Knoten $s, t$ gilt: der Pfad von $s$ nach $t$ mit den wenigsten Streckenkilometern entspricht dem Pfad mit dem minimalem Energieverbrauch.
  \end{enumerate}

  \begin{solution}

  \end{solution}
\end{exercise}

\begin{exercise}{NP-Vollständigkeit}
  Paul hat einen Kasten mit $N$ Bauklötzen, die Klötze $k_i$ haben jeweils die Maße $b_i, t_i$ und $h_i$ (Breite / Tiefe / Höhe). Paul fragt sich, ob er mit den Klötzen einen Turm bauen kann, der exakt genau so hoch ist wie sein Schreibtisch (Schreibtischhöhe $H$). Dabei kann jeder Klotz beliebig gestapelt werden.
  \begin{enumerate}
    \item Zeigen Sie durch Reduktion auf ein Ihnen bekanntes Problem, dass Pauls Problem NP-vollständig ist.
    \item Entwickeln Sie ein Backtracking-Algorithmus zur Lösung von Pauls Problem. Skizzieren Sie in wenigen Stichpunkten Ihren Lösungsweg und geben Sie den Algorithmus in Pseudo-Code an.
    \item Paul möchte nun den höchsten Turm bauen, der noch unter seinen Schreibtisch passt. Geben Sie Schranken an, die verwendet werden können um den Backtracking-Algorithmus in einen Branch\&Bound Algorithmus umzuwandeln.
    \item Paul stellt beim Vermessen seiner $N$ Klötze fest, dass alle eine Grundfläche von $1 \times 1$ cm haben und außerdem entweder $6, 3$ oder $2$ cm hoch sind. Ist Pauls Problem unter dieser Randbedingung noch NP-vollständig? Begründen Sie Ihre Antwort.
  \end{enumerate}

  \begin{solution}

  \end{solution}
\end{exercise}

\end{document}