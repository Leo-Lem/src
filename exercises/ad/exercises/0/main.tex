\documentclass{article}

\usepackage{exercises}
\withsolutions

\begin{document}
\sheet[0]{Grundlagen}
\begin{eexercises}{Vollständige Induktion}{
    Beweisen Sie die folgenden Aussagen mit vollständiger Induktion:
  }
  \item $\sum_{k=1}^{n}{k} = \frac{n(n+1)}{2}\forall n \in \mathbb{Z}_{\ge 0}$.
  \item $\sum_{k=0}^{n-1} 2^k = 2^n-1\forall n \in \mathbb{Z}_{\ge 0}$.
\end{eexercises}

\begin{solutions}
  \item \induction{
    \sum_{k=1}^{0}{k} = 0 = \frac{0(0+1)}{2}
  }{
    \sum_{k=1}^{n}{k} = \frac{n(n+1)}{2}
  }{
    \sum_{k=1}^{n+1}{k} = \sum_{k=1}^{n}{k} + (n+1) = \frac{n(n+1)}{2} + (n+1) = \frac{n(n+1)+2(n+1)}{2} = \frac{(n+1)(n+2)}{2}
  }
  \item \induction{
    \sum_{k=0}^{0-1} 2^k = 0 = 2^0-1
  }{
    \sum_{k=0}^{n-1} 2^k = 2^n-1
  }{
    \sum_{k=0}^{n} 2^k = 2^n-1 + 2^n = 2^{n+1}-1
  }
\end{solutions}

\begin{exercise}{Algorithmenanalyse}
  Was berechnet der Algorithmus Alg? Finden Sie einen Aufruf für den der Algorithmus besonders langsam ist. Wie verhält sich der Algorithmus bei Eingabe von rationalen oder reellen Zahlen?
  \documentclass{article}

\usepackage{exercises}

\begin{document}
\begin{algorithm}[ht]
  \caption{Alg}

  \KwData{$a,b \in \mathbb{Z}_{>0}$}
  \KwResult{$c \in \mathbb{Z}_{>0}$}
  \BlankLine

  \lIf{$a=b$}{\Return{$a$}}
  \lIf{$a<b$}{\Return{$\text{Alg}(b-a,a)$}}
  \lIf{$b<a$}{\Return{$\text{Alg}(a-b,b)$}}
\end{algorithm}
\end{document}

  \begin{solution}
    Der Algorithmus berechnet den größten gemeinsamen Teiler von $a$ und $b$. Der Algorithmus ist besonders langsam, wenn $a$ und $b$ sehr unterschiedlich groß sind. Da u. U. kein gemeinsamer Teiler existiert, kann es sein, dass der Algorithmus nicht terminiert.
  \end{solution}
\end{exercise}

\begin{exercise}{Algorithmenentwurf}
  Entwerfen Sie einen Algorithmus, der für gegebene ganze Zahlen $a_1, \ldots, a_n \in \mathbb{Z}$ das Minimum und das Maximum bestimmt. Versuchen Sie dabei möglichst wenige Vergleiche zu verwenden ($\leq 1,5n$ sind möglich).

  \begin{solution}
    \documentclass{article}

\usepackage{exercises}

\begin{document}
\begin{algorithm}[ht]
  \caption{MinMax mit Pairwise Comparison}\label{alg:minMaxPairwise}

  \KwData{$a_1, \ldots, a_n \in \mathbb{Z}$}
  \KwResult{$\min, \max \in \mathbb{Z}$}
  \BlankLine

  $\min \gets a_1$
  $\max \gets a_1$

  \For{$i \gets 2$ to $n$ in Schritten von 2}{
    \If{$i = n$}{
      \If{$a_i < \min$}{$\min \gets a_i$}
      \If{$a_i > \max$}{$\max \gets a_i$}
    }
    \Else{
      \If{$a_i < a_{i+1}$}{
        $\min' \gets a_i$;
        $\max' \gets a_{i+1}$
      }
      \Else{
        $\min' \gets a_{i+1}$;
        $\max' \gets a_i$
      }

      \If{$\min' < \min$}{$\min \gets \min'$}
      \If{$\max' > \max$}{$\max \gets \max'$}
    }
  }
\end{algorithm}
\end{document}
  \end{solution}
\end{exercise}
\end{document}