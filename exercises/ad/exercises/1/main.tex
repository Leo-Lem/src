\documentclass{article}

\usepackage[solutions]{xrcise}

\begin{document}
\sheet{Big O \& Schleifeninvarianten}
\begin{exercise}{Big-O Notation}
  Beweisen oder widerlegen Sie die folgenden Aussagen:
  \begin{enumerate}
    \item $2n = \bigO(n)$
    \item $\frac{1}{n} = \bigO(1)$
    \item $n = \bigO(1)$
    \item $n^2 = \bigO(n)$
    \item $n \log n = \bigO(n^2)$
    \item $n^2 + n = \bigO(n^2)$
  \end{enumerate}

  \begin{solution}
    \begin{enumerate}
      \item Gilt für $c = 2$ und $n_0 = 1$.
      \item Gilt für $c = 1$ und $n_0 = 1$.
      \item Da $\lim_{n\to\infty} \frac{n}{1} = \infty \not\le c$ ist $n \neq \bigO(1)$.
      \item Da $\lim_{n\to\infty} \frac{n^2}{n} = \infty \not\le c$ ist $n^2 \neq \bigO(n)$.
      \item Gilt für $c = 1$ und $n_0 = 1$.
      \item Gilt für $c = 2$ und $n_0 = 1$.
    \end{enumerate}
  \end{solution}
\end{exercise}

\begin{exercise}{Algorithmus BALLSELECTION}
  Betrachten Sie den folgenden, umgangssprachlich formulierten Algorithmus. Gegeben sei eine Urne mit einer geraden Anzahl $n = 2k \ge 1$ weißer Kugeln und einer beliebigen Anzahl $m \geq 1$ schwarzer Kugeln.\par
  \begin{algorithm}[ht]
    \caption{BALLSELECTION}
    \While{wenigstens zwei Kugeln in der Urne sind}{
      Nimm zwei beliebige Kugeln aus der Urne heraus.\\
      \If{beide Kugeln die gleiche Farbe haben}{
        Entferne die Kugeln aus dem Spiel und lege eine neue, schwarze Kugel in die Urne.
      }
      \Else{
        Lege die weiße Kugel in die Urne zurück und entferne lediglich die schwarze Kugel aus dem Spiel.
      }
    }
  \end{algorithm}
  \noindent Beweisen Sie, dass die letzte Kugel in der Urne schwarz ist. Geben Sie dazu eine geeignete Schleifeninvariante an.

  \begin{solution}
    Schleifeninvariante: $n$ ist gerade. Terminationsszenarien:
    \begin{itemize}
      \item Zwei weisse Kugeln in der Urne: Beide werden gezogen, eine schwarze hineingelegt und die Schleife terminiert.\par $\hookrightarrow$ Die letzte Kugel ist schwarz.
      \item Es sind nur noch schwarze Kugeln in der Urne $\to$ Die letzte Kugel ist schwarz, da keine weisse Kugel mehr hinzugefügt wird.
    \end{itemize}
  \end{solution}
\end{exercise}

\begin{exercise}{Eigenschaften der Big-O Notation}
  Sei F die Menge der Funktionen $\mathbb{N} \rightarrow \mathbb{R}_{>0}$ und seien $e, f, g, h \in F$. Beweisen oder widerlegen Sie die folgenden Aussagen:
  \begin{enumerate}
    \item Aus $e = \bigO(f)$ und $g = \bigO(h)$ folgt $(e \cdot g) = \bigO(f \cdot h)$.
    \item Aus $e = \bigO(f)$ und $g = \bigO(h)$ folgt $(e + g) = \bigO(f + h)$.
    \item Aus $e = \bigO(f)$ und $g = \bigO(h)$ folgt $(e + g) = \bigO(\max(f,h))$.
  \end{enumerate}
  \hint{$\cdot, +, \max : F \times F \rightarrow F$ sind punktweise aufzufassen, also ist zum Beispiel $(e \cdot g)$ die durch $(e \cdot g)(n) = e(n) \cdot g(n)$ definierte Funktion.}

  \begin{solution}
    \begin{enumerate}
      \item $\forall n>\max(n_1,n_2)$ gilt:
            \begin{align*}
              e \cdot g & \leq c_1 f \cdot c_2 h \quad|\ c=c_1c_2 \\
              e \cdot g & \leq c (f \cdot h)
            \end{align*}
      \item $\forall n>\max(n_1,n_2)$ gilt:
            \begin{align*}
              e + g & \leq c_1 f + c_2 h \quad|\ c_1 \text{ und } c_2 \leq c=c_1+c_2 \\
              e + g & \leq cf + ch                                                   \\
              e + g & \leq c(f + h)
            \end{align*}
      \item $\forall n>\max(n_1,n_2)$ gilt:
            \begin{align*}
              e + g & \leq c_1 \cdot f + c_2 \cdot h \quad|\ f\text{ und } h \leq max(f,h) \\
              e + g & \leq c_1 \cdot \max(f,h) + c_2 \cdot \max(f,h)                       \\
              e + g & \leq c \cdot \max(f,h)
            \end{align*}
    \end{enumerate}
  \end{solution}
\end{exercise}

\begin{exercise}{Größenordnung}
  Ordnen Sie die über die folgenden Ausdrücke definierten Funktionen $f_i : \mathbb{N}_{>0} \rightarrow \mathbb{R}_{\geq 0}, i = 1, \ldots, 10$ der Größe nach im Sinne der $\bigO$-Notation:
  \begin{table}[ht]
  \centering
  \begin{tabular}[ht]{c|c|c|c|c | c|c|c|c|c}
    $f_1(n)$  & $f_2(n)$      & $f_3(n)$ & $f_4(n)$   & $f_5(n)$    & $f_6(n)$ & $f_7(n)$    & $f_8(n)$ & $f_9(n)$                                     & $f_{10}(n)$ \\
    \hline
    $n^{3/2}$ & $\log\log{n}$ & $n^n$    & $\sqrt{n}$ & $n \log{n}$ & $n$      & $\log^2{n}$ & $2^n$    & $n^{\epsilon}\forall 0<\epsilon<\frac{1}{2}$ & $\log{n}$
  \end{tabular}

  \caption{Funktionen $f_i$}
\end{table}

  \begin{solution}
    \begin{equation*}
      f_2 < f_{10} < f_7 < f_9 < f_4 < f_6 < f_5 < f_1 < f_8 < f_3
    \end{equation*}
  \end{solution}
\end{exercise}

\begin{exercise}{Big-O Notation}
  Beweisen oder widerlegen Sie die folgenden Aussagen:
  \begin{enumerate}
    \item $n^3 = \bigO(3n^2)$
    \item $n + \log(n) = \bigO(n \cdot \log(n))$
    \item $n^5 = \bigO(2^{log^2{n}})$
    \item $3^n = \bigO(2^n)$
  \end{enumerate}

  \begin{solution}
    \begin{enumerate}
      \item Gilt nicht, da $\lim_{n\to\infty} \frac{n^3}{3n^2} = \infty \not\le c$.
      \item Gilt für $c = 2$ und $n_0 = 1$.
      \item Gilt, für $c = 1$ und $n_0 = 2^5 = 32$.
      \item Gilt nicht, da $\lim_{n\to\infty} \frac{3^n}{2^n} = \infty \not\le c$.
    \end{enumerate}
  \end{solution}
\end{exercise}

\begin{exercise}{Schleifeninvarianten}
  Die folgende Funktion erhält ein Eingabearray aus natürlichen Zahlen als Eingabe.
  \documentclass{article}

\usepackage{exercises}

\begin{document}
\begin{algorithm}[ht]
  \caption{Alg}\label{alg:avg}

  \KwData{$A[i], i \in [1,n], A[i] \in \mathbb{N}$}
  \KwResult{$\text{result} \in \mathbb{R}_{>0}$}
  \BlankLine

  $\text{result} \gets 0$ \\
  \For{$i = 1$ to $n$}{
    $\text{result} \gets ((i - 1) \cdot \text{result} + A[i])/i$
  }
  \Return{$\text{result}$}
\end{algorithm}
\end{document}
  \begin{enumerate}
    \item Wenden Sie den Algorithmus auf ein Beispielarray der Länge mindestens drei an.
    \item Geben Sie an, was der Algorithmus berechnet.
    \item Bestimmen Sie eine geeignete Schleifeninvariante.
    \item Beweisen Sie mit Hilfe der Schleifeninvariante die Korrektheit des Algorithmus.
  \end{enumerate}

  \begin{solution}
    \begin{enumerate}
      \item Beispielarray: $A = [3,10,11]$:
            \begin{align*}
              \text{result} & = 0                                   \\
              \text{result} & = ((1-1) \cdot 0 + 3)/1 = 3/1 = 3     \\
              \text{result} & = ((2-1) \cdot 3 + 10)/2 = 13/2 = 6.5 \\
              \text{result} & = ((3-1) \cdot 6.5 + 11)/3 = 24/3 = 8
            \end{align*}
      \item Der Algorithmus berechnet den Durchschnitt der Elemente im Array.
      \item Schleifeninvariante: $\text{result} = \sum_{k=1}^{i} A[k]/i$.
            \invariant{
              In erster Iteration ist die Invariante trivialerweise korrekt, da Durchschnitt von einem Element gleich dem Element ist.
              \begin{align*}
                \text{result} & = ((1-1)\cdot 0 + A[1])/1 \\
                              & = A[1]                    \\
                              & = \sum_{k=1}^{1} A[k]/1
              \end{align*}
            }{
              Wird eine korrekte Schleifeninvariante vor Iteration $i$ angenommen, ist die Invariante danach ebenfalls korrekt.
              \begin{align*}
                \text{result} & = ((i-1)\cdot \text{result} + A[i-1])/(i-1)        \\
                              & = ((i-1)\cdot\sum_{k=1}^{i-1} A[k]/(i-1) + A[i])/i \\
                              & = (\sum_{k=1}^{i-1} A[k] + A[i])/i                 \\
                              & = \sum_{k=1}^{i} A[k]/i
              \end{align*}
            }{
              Die Korrektheit bei Termination leitet sich aus der Korrektheit in der Erhaltung ab.
              \begin{align*}
                \text{result} & = ((n-1)\cdot \text{result} + A[n-1])/(n-1)        \\
                              & = ((n-1)\cdot\sum_{k=1}^{n-1} A[k]/(n-1) + A[n])/n \\
                              & = (\sum_{k=1}^{n-1} A[k] + A[n])/n                 \\
                              & = \sum_{k=1}^{n} A[k]/n
              \end{align*}
            }
    \end{enumerate}
  \end{solution}
\end{exercise}


\end{document}